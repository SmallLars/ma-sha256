\chapter{CD und Inhalt}

\begin{figure}[!ht]
\centering
\framebox[13cm][c]{\vbox to 13.4cm{\vfil CD \vfil}}
\caption{CD}
\label{tbl:a-cd}
\end{figure}

\clearpage

Im Folgenden werden die Ordner der CD, sowie deren Inhalte, in der obersten Ebene beschrieben. \newline

\begin{itemize}
  \setlength{\itemsep}{0.5cm}
  \item \textbf{blaster}: Programm blaster, das die Server-Software vor dem Flashen mit \newline gerätespezifischen Informationen ergänzt.
  \item \textbf{border-router}: Aus Contiki entnommener Border-Router, der mit der \newline IP-Adresse aaaa::60b1:60b1:60b1:0022 konfiguriert ist.
  \item \textbf{client}: Client inklusive libcoap und \acr{dtls}.
  \item \textbf{contiki}: SmartAppContiki mit zusätzlichen Apps, wie \acr{dtls}, \acr{ecc} und flash-store.
  \item \textbf{dokumente}: Einige in dieser Arbeit genutzte Dokumente und Software, inklusive Quellenangaben.
  \item \textbf{expose}: Exposé zur Bachelorarbeit.
  \item \textbf{kolloquien}: Präsentationen der beiden Kolloquien, in denen die Bachelorarbeit vorgestellt wurde.
  \item \textbf{libmc1322x}: Bibliothek zur Nutzung des \glos{mc1322}, die in dieser Arbeit erweitert wurde.
  \item \textbf{thesis}: Quellcode dieses Dokuments inklusive der erzeugten PDF-Datei.
  \item \textbf{server}: Server mit einigen Ressourcen, der auf Contiki basiert. \newline Vorkonfiguriert mit der IP-Adresse aaaa::60b1:60b1:60b1:0019.
  \item \textbf{server-min}: Minimaler \acr{coap}-Server mit einer Ressource und dem unveränderten Contiki.
  \item \textbf{server-tiny}: \acr{dtls}-Server mit TinyDTLS. Zu groß für den \glos{mc1322}, \newline aber notwendig für den Größenvergleich einiger Komponenten.
  \item \textbf{sniffer}: Aus Contiki entnommener Sniffer, der mit der IP-Adresse aaaa::60b1:60b1:60b1:0028 \newline konfiguriert ist. Außerdem sind einige Mitschnitte von \glospl{handshake} enthalten.
  \item \textbf{windows}: Notwendige Treiber und ein Script, um die Datenausgabe des Servers \newline per USB auch in Windows abzurufen.
  \item \textbf{wireshark}: Wireshark-Dissector für das in dieser Arbeit entwickelte Protokoll, \newline mit Wireshark in der genutzten Version.
\end{itemize}
