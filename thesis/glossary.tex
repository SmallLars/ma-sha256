
\newglossaryentry{glos:handshake}{
  name={Handshake},
  plural={Handshakes},
  description={Beschreibt die Aushandlung von Sicherheitsparametern um eine sichere Verbindung herzustellen}
}

\newglossaryentry{glos:ciphersuite}{
  name={Cipher-Suite},
  plural={Cipher-Suites},
  description={Gruppe aus 4 Algorithmen, die für Schlüsselaustausch, Authentifizierung, Hash und Verschlüsselung verwendet werden}
}

\newglossaryentry{glos:fragmentierung}{
  name={Fragmentierung},
  description={Beschreibt die Aufteilung eines Ganzen in kleinere Teile}
}

\newglossaryentry{glos:fragment}{
  name={Fragment},
  plural={Fragmente},
  description={Beschreibt einen Teil eines Ganzen}
}

\newglossaryentry{glos:rsa}{
  name={RSA-Verfahren},
  description={Asymmetrisches kryptographisches Verfahren zur Verschlüsselung und Signatur, das nach seinen Erfindern Rivest, Shamir und Adleman benannt ist}
}

\newglossaryentry{glos:mitma}{
  name={Man-in-the-middle-Angriff},
  description={Angriff auf eine Kommunikationsverbindung zwischen zwei Parteien, bei dem ein Angreifer die vollständige Kontrolle über den Datenverkehr der Verbindung übernimmt}
}

\newglossaryentry{glos:mc1322}{
  name={MC13224v},
  description={ARM7TDMI-S Microcontroller der Firma Freescale Semiconductor, Inc}
}

\newglossaryentry{glos:gobi}{
  name={GOBI},
  description={Name eines studentischen Projekts an der Universität Bremen, das im Jahr 2012 als Bachelorprojekt begonnen hat und im Jahr 2013 als Masterprojekt forgesetzt wird}
}

% \newglossaryentry{label}{
%   name={name},
%   description={long description}
% }