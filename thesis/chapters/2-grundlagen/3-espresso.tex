\section{Espresso}
\label{sec:espresso}

Espresso ist ein heuristischer Logikminimierer der Berkeley Universität von 1989. Verwendet wird in dieser
Arbeit eine moderne Version von 2012 \cite{espressobin}. Als Eingabe dienen Wahrheitstabellen wie sie
unter anderem mit "`eqntott"' (siehe Abschnitt \ref{sec:knf}) generiert werden. Die Ausgabe ist eine
Wahrheitstabelle in minimierter Form.

Als Parameter werden "`-Dexact"' und "`-epos"' verwendet. -Dexact garantiert eine minimale Anzahl von
Einträgen in der Wahrheitstabelle und somit von Klauseln in der konjunktiven Normalform. Die Laufzeit
ist dabei exponentiell im Bezug zur Anzahl der Variablen. Da Espresso in dieser Arbeit nur für
Wahrheitstabellen mit bis zu 14 Variablen verwendet wird, sind Lösungen noch innerhalb eines Tages möglich.
-epos ist notwendig, um die Menge der wahren (On-Set) und nicht-wahren Zeilen (Off-Set) der Wahrheitstabelle
zu invertieren. Ohne die Angabe des Parameters minimiert Espresso die Menge der wahren Zeilen. Benötigt
wird für die konjunktive Normalform jedoch eine minimale Menge der nicht-wahren Zeilen.