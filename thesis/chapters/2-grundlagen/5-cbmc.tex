\section{C Bounded Model Checking}
\label{sec:cbmc}

\acr{cbmc} ist ein Model Checker für C und C++ Programme. Verwendung findet \acr{cbmc} in dieser Arbeit in der Evaluation in Kapitel \ref{chp:evaluation}.
Um einen Beweis zu führen, konvertiert \acr{cbmc} Programmcode automatisch in eine konjunktive Normalform und übergibt diese an einen SAT-Solver.
Relevant sind dafür in dieser Arbeit zwei Funktionen mit denen sich ein Beweis führen lässt. Mit nondet\_uint() können ganzzahligen Variablen markiert
werden, denen \acr{cbmc} beliebige Werte zuweisen darf. Innerhalb von assert() können Bedingungen formuliert werden, die \acr{cbmc} beweisen soll.

Bei dem Beispiel des Addierers (wie in Abschnitt \ref{sec:satsolver}) könnten die beiden Summanden beliebige Werte annehmen. Um zwei Summanden für die
Summe 42 zu ermitteln, könnte die Bedingungen im assert() enthalten sein, dass die Summe immer ungleich 42 sein muss. Das veranlasst \acr{cbmc} dazu, nach
zwei Summanden zu suchen, die zusammen 42 ergeben. Gelingt dies, ist der Beweis fehlgeschlagen und \acr{cbmc} gibt die gefundenen Summanden als Beispiel aus.