\section{SAT-Solver}
\label{sec:satsolver}

SAT-Solver dienen dazu, ein Erfüllbarkeitsproblem der Aussagenlogik zu lösen. Als Eingabe wird eine Formel in konjunktiver Normalform genutzt.
Ist die Formel erfüllbar, liefert ein SAT-Solver dafür ein Beispiel. Da die Laufzeit exponentiell zur Anzahl der Variablen ist, gelingt
es jedoch nicht immer eine Lösung zu finden oder die Unerfüllbarkeit zu belegen. Relevant für diese Arbeit ist, dass ein SAT-Solver bei einem
Lösungsversuch Konfliktklauseln sammelt. Diese ergeben sich aus einem Versuch mehrere Variablen zu belegen der sich als ungültig herausstellt.
Konfliktklauseln ergänzen die ursprüngliche konjunktive Normalform. Anstatt diese zu verwerfen, werden sie in dieser Arbeit exportiert, gefiltert
und in weitere Lösungsversuche eingebracht um den Lösungsprozess zu beschleunigen.

Beachtet werden muss dabei, dass die Konfliktklauseln sich auf eine spezifische Lösung beziehen. Soll ein SAT-Solver zum Beispiel zwei Summanden
einer Summe finden, wird er mit den Konfliktklauseln eines erfolgreichen Lösungsversuchs immer wieder zum gleichen Ergebnis kommen, obwohl es
unter Umständen viele weitere Lösungen gibt, da diese durch die Konfliktklauseln ausgeschlossen werden. Vermeiden lässt sich dieses Verhalten,
wenn ausschließlich die Logik (in diesem Fall die Addition) als Eingabe dient. Spezifische Werte wie der Summand können als Annahmen (Assumptions)
in den Lösungsprozess aufgenommen werden. Konfliktklauseln die durch diese Annahmen entstehen werden separat behandelt und bei Programmende verworfen.

In dieser Arbeit wird CryptoMiniSat in der Version 4.5.3 verwendet. CryptoMiniSat bietet den Vorteil auch Klauseln einlesen zu können, deren Literale
durch den XOR-Operator verknüpft sind. Dies ermöglicht eine kompaktere Repräsentation der Eingabe, da eine XOR-Klausel mit $n$ Literalen $2^{n-1}$
normale Klauseln repräsentiert. Ein weiterer Vorteil von CryptoMiniSat ist, dass die ursprüngliche konjunktive Normalform während eines Lösungsversuchs
optimiert wird. Diese optimierte Form kann als irredundante Klauselmenge genau wie die Konfliktklauseln (redundante Klauselmenge) exportiert werden.