\section{Majority (MAJ)}
\label{sec:knf:maj}

Die Majority-Funktion ist ähnlich aufgebaut wie die Choose-Funktion und enthält ein AND-Gatter mehr. Für die auf den Gattern basierte Lösung sind somit drei zusätzliche
Literale notwendig. Anstatt vier werden sieben Literale benötigt. Für die AND-Gatter werden jeweils drei Klauseln benötigt. Für die XOR- bzw. OR-Funktion wird jeweils
ein Gatter mit drei Eingängen herangezogen um zumindest hier ein zusätzliches Literals zu vermeiden. Das XOR-Gatter benötigt eine XOR-Klausel oder acht normale Klauseln
während das OR-Gatter vier Klauseln benötigt. Je nach Realisierung ergeben sich damit siebzehn, dreizehn oder zehn Klauseln.

Gegenübergestellt wird die Lösung von eqntott und Espresso die sich aus der boolschen Gleichung in Abbildung \ref{fig:majority_eqn} ergibt.
\begin{figure}[!h]
  \centering
  \begin{lstlisting}[]
  NAME = MAJ;
  INORDER = r_out a_in b_in c_in;
  OUTORDER = z;

  z = eq(r_out, xor(xor(a_in & b_in, a_in & c_in), b_in & c_in));
  \end{lstlisting}
  \caption{Majority - Gleichung}
  \label{fig:majority_eqn}
\end{figure}

Die ermittelte konjunktive Normalform ist in Abbildung \ref{fig:majority_cnf} aufgeführt.
Sechs Klauseln reichen aus um die Majority-Funktion zu beschreiben.
\begin{figure}[!h]
  \centering
  \begin{minipage}[l]{2cm}
    $ (r \vee \overline{a} \vee \overline{b}) ~ \wedge $\\
    $ (\overline{r} \vee a \vee b) ~ \wedge $\\
    $ (r \vee \overline{a} \vee \overline{c}) ~ \wedge $\\
    $ (r \vee \overline{b} \vee \overline{c}) ~ \wedge $\\
    $ (\overline{r} \vee a \vee c) ~ \wedge $\\
    $ (\overline{r} \vee b \vee c) $
  \end{minipage}
  \caption{Majority - Konjuktive Normalform}
  \label{fig:majority_cnf}
\end{figure}

In Abbildung \ref{fig:majority_literalclausecount} sind die unterschiedlichen Lösungen gegenübergestellt.
Genau wie die Choose-Funktion wird die Majority-Funktion bitweise auf einer 32 Bit breiten Binärzahl angewendet. Jedes Bit der Eingabe wird dabei nur einmal genutzt.
Die einzelnen Berechnungen sind somit unabhängig wodurch sich die Gesamtanzahl der Literale und Klauseln aus der Multiplikation mit 32 ergibt.
\begin{figure}[!h]
  \centering
  \begin{tabular}{l|r|r}
    \hiderowcolors
                           & Majority &    Gesamt \\
    \hline
    Gatter                 &  7 -  17 & 224 - 544 \\
    Gatter mit OR          &  7 -  13 & 224 - 416 \\
    Gatter mit XOR         &  7 -  10 & 224 - 320 \\
    eqntott \& Espresso    &  4 - ~~6 & 128 - 192 \\
    \showrowcolors
  \end{tabular}
  \caption{Majority - Literale und Klauseln}
  \label{fig:majority_literalclausecount}
\end{figure}

Auch dieses Ergebnis zeigt, dass es nicht notwendig ist, unterschiedliche Lösungen bereitzustellen.
Egal ob eine XOR-Unterstützung gegeben ist oder nicht, ist die von eqntott und Espresso ermittelte Lösung optimal und wird implementiert.