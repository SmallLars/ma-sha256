\chapter{Erzeugung der konjunktiven Normalform}
\label{chp:knf}

Der einfachste Weg eine konjunktive Normalform zu erzeugen, die erfüllbarkeitsäquivalent zu \glos{sha256} ist, führt vermutlich über die Verwendung vom \acr{cbmc}
(siehe Abschnitt \ref{sec:cbmc}). \acr{cbmc} erzeugt zur Verifikation von Aussagen über C-Programmcode eine konjunktive Normalform und übergibt diese direkt
einem SAT-Solver (siehe Abschnitt \ref{sec:satsolver}). Es besteht jedoch auch die Möglichkeit die generierte konjunktive Normalform im DIMACS-Format auszugeben,
so dass ein C-Programm automatisch übersetzt werden kann. Dabei gehen jedoch jegliche Informationen über den Aufbau und die Bedeutung einzelner Literale verloren,
so dass es nicht möglich ist, erworbenes Wissen über Literale einzelnen Berechnungen zuzuordnen. Ausgehend von der Addition, die in der Kompressionsfunktion von
\glos{sha256} am häufigsten verwendet wird, ist es somit nicht möglich, erworbenes Wissen darüber zuzuordnen und auf alle weiteren Additionen zu übertragen.
Außerdem ist es nicht möglich Einfluss auf die Anzahl und Verwendung der Literale zu nehmen. Für einen Addierer liegt die Entscheidung somit bei \acr{cbmc},
wie dieser realisiert wird und ob und wie viele zusätzliche Literale verwendet werden. Das eine Extrem wäre die Verwendung eines Carry-Ripple-Addieres, dessen einzelne
Volladdierer in Gatter zerlegt werden, die dann einzeln in die konjunktive Normalform überführt werden. Das führt zu vergleichsweise vielen Literalen mit wenig
kurzen Klauseln. Das andere Extrem wäre der Versuch, eine konjunktive Normalform zu erzeugen, die ausschließlich Literale für die Summanden und die Summe erzeugt.
Dabei entstehen jedoch vergleichsweise viele lange Klauseln.

Um sowohl Kontrolle über die Erzeugung der konjunktiven Normalform zu bekommen als auch Informationen zu sammeln um eine Analyse zu ermöglichen, wird ein Programm
erstellt, dass das Entwurfsmuster "`Besucher"' \cite[301]{visitor} verwendet. Besucher sind dabei Instanzen von Klassen, die unterschiedlichste Aufgaben erfüllen können.
Eine Aufgabe kann es dabei sein, die konjunktive Normalform zu erzeugen während eine andere Aufgabe das Zählen von Literalen und Klauseln sein kann.
Besucht wird dabei eine Struktur, deren Objekte das Verhalten von \glos{sha256} beschreiben. Die Objekte der Struktur werden im folgenden als Modul bezeichnet.
Ein Modul kann sowohl die vollständige Kompressionsfunktion von \glos{sha256} sein, als auch ein kleiner Baustein wie ein Halbaddierer. Dabei kann ein Modul
auch andere Module verwenden. Es zeigt ich, dass acht grundlegende Module ausreichen um \glos{sha256} vollständig zu beschreiben. Diese werden in den Abschnitten
\ref{sec:knf:addierer} bis \ref{sec:knf:sig} erläutert. Alle weiteren Module setzen sich aus diesen zusammen und werden in Abschnitt \ref{sec:knf:module} erläutert.

Alle notwendigen allgemeinen Funktionen für ein Modul sind in einer Basisklasse hinterlegt, von der jedes konkrete Modul erben muss. Ein konkretes Modul wird
dadurch realisiert, dass die bis dato virtuelle Funktion create(Printer* printer) implementiert wird.



In der Implementierung wird der Besucher printer genannt, weil der erste besucher ein dimacsprinter war.



modul:\\
void create(Printer* printer)\\
~\\
besucher:\\
void newModul(unsigned level, const char* name, Modul* modul)\\
void create(bool xOR, const std::vector<CMSat::Lit>\& vars)





Ziele:  nicht nur knf generieren\\
auch wissen darüber sammeln\\
besucher (printer) benötigt methode um klauseln zu sammeln.
~\\
besucherhat 2 funktionen:\\
klauseln sammeln und informationen sammeln\\
~\\
support für mit und ohne xor\\
immer xor nutzen: bei bedarf rückwandeln: einfacher und compakter als andersrum\\
klasueln sammeln und analysieren ist aufwendig
~\\
hilfsmittel clauseprinter:\\
ermöglicht: literale brauchen nur einmal eingebeben werden:\\
werden durch konkrete belegung erzeugt\\
~\\
baum\\
hierarchie durch level:\\
subaddierer bekommen ihre ebene: 1 bis 4\\
module aus grafik ebene 10\\
11\\
20\\
21\\
sha als ganzes: 30\\
~\\
betrachtung der erweiterung als teil der runden 17 bis 64\\
~\\
moduletests\\
https://github.com/siu/minunit mit minimaler anpassung für c++

\section{Gatter}
\label{sec:knf:gatter}

Grundlage für die Tseitin-Transformation (siehe Abschnitt \ref{sec:knf}) sollen in dieser Arbeit die Operatoren AND, OR und XOR bilden. Verwendet werden sie jedoch
nur, falls sie in einem Modul eine direkte Beziehung zwischen Eingangs- und Ausgang-Literalen definieren, weil sonst zusätzliche Literale eingefügt werden müssen.
Die Boolesche Gleichungen für die drei Operatoren sind in Abbildung \ref{fig:gatter_equations} dargestellt. a und b werden als Literale für die Eingänge verwendet
und r steht für den Ausgang (das Resultat).

\begin{figure}[!h]
  \centering
  \begin{minipage}[c]{4.85cm}
    \begin{lstlisting}[]
  NAME = AND;
  INORDER = r_out a_in b_in;
  OUTORDER = z;

  z = eq(r_out, (a_in & b_in));
    \end{lstlisting}
  \end{minipage}
  \begin{minipage}[c]{4.85cm}
    \begin{lstlisting}[]
  NAME = OR;
  INORDER = r_out a_in b_in;
  OUTORDER = z;

  z = eq(r_out, (a_in | b_in));
    \end{lstlisting}
  \end{minipage}
  \begin{minipage}[c]{5.1cm}
    \begin{lstlisting}[]
  NAME = XOR;
  INORDER = r_out a_in b_in;
  OUTORDER = z;

  z = eq(r_out, xor(a_in, b_in));
    \end{lstlisting}
  \end{minipage}
  \caption{Gatter - Gleichungen}
  \label{fig:gatter_equations}
\end{figure}

Nach der Verwendung von eqntott und Espresso ergeben sich aus den Booleschen Gleichungen die konjunktiven Normalformen in Abbildung \ref{fig:gatter_cnf}.
Diese decken sich mit den Angaben aus "`Digitaltechnik - Eine praxisnahe Einführung"' \cite[164]{digitaltechnik}. Für die Negation der Operatoren reicht es aus, das Ergebnisliteral r zu invertieren.

\begin{figure}[!h]
  \centering
  \begin{minipage}[l]{4.65cm}
    \underline{AND}\\
    $ (\overline{r} \vee a) ~ \wedge $\\
    $ (\overline{r} \vee b) ~ \wedge $\\
    $ (r \vee \overline{a} \vee \overline{b}) $\\
    ~
  \end{minipage}
  \begin{minipage}[l]{4.65cm}
    \underline{OR}\\
    $ (r \vee \overline{a}) ~ \wedge $\\
    $ (r \vee \overline{b}) ~ \wedge $\\
    $ (\overline{r} \vee a \vee b) $\\
    ~
  \end{minipage}
  \begin{minipage}[l]{4.3cm}
    \underline{XOR}\\
    $ (\overline{r} \vee \overline{a} \vee \overline{b}) ~ \wedge $\\
    $ (r \vee a \vee \overline{b}) ~ \wedge $\\
    $ (r \vee \overline{a} \vee b) ~ \wedge $\\
    $ (\overline{r} \vee a \vee b) $
  \end{minipage}
  \caption{Gatter - Konjunktive Normalformen}
  \label{fig:gatter_cnf}
\end{figure}

Während es bei den Operatoren AND und OR gelungen ist, Don't-Care Literale zu identifizieren und für eine Vereinfachung zu nutzen, ist dies bei dem XOR-Operator
nicht möglich. Jede einzelne Änderung eines beliebigen Eingangsliterals führt zu einer Änderung des Ausgangssignals. Genau die Hälfte der möglichen Belegungen
ist somit nicht erfüllbar und muss in die konjunktive Normalform mit aufgenommen werden. Allgemein gilt, dass bei einem XOR mit n Eingängen $ 2^{n} $ Klauseln
mit jeweils $ (n + 1) $ Literalen notwendig sind. Die Klauselmenge wächst exponentiell im Bezug zur Anzahl der Eingänge.

CryptoMiniSat (siehe Abschnitt \ref{sec:satsolver}) ist in der Lage, neben normalen Klauseln (Disjunktion) auch Klauseln zu berücksichtigen, deren Literale
mit dem XOR-Operator verknüpft sind. Genau wie normale Klauseln müssen sie für die Erfüllbarkeit zu $1$ evaluiert werden. Damit lässt sich das exponentielle
Wachstum der Klauselmenge umgehen. Für ein XOR mit n Eingängen ist nur noch eine Klausel mit $ (n + 1) $ Literalen notwendig. Abbildung \ref{fig:gatter_cnf_xor}
zeigt die Klausel für den XOR-Operator mit zwei und drei Eingängen.
\begin{figure}[!h]
  \centering
  \begin{minipage}[l]{1cm}
    ~\\
    ~
  \end{minipage}
  \begin{minipage}[l]{2.5cm}
    Zwei Eingänge:\\
    Drei Eingänge:
  \end{minipage}
  \begin{minipage}[l]{3cm}
    $ (\overline{r} \veebar a \veebar b) $\\
    $ (\overline{r} \veebar a \veebar b \veebar c) $
  \end{minipage}
  \caption{Gatter - XOR}
  \label{fig:gatter_cnf_xor}
\end{figure}

Abschließend wird noch ein Operator für die Negation (NOT) benötigt. Um daraus einen Gleichheits-Operator (EQ) zu machen, reicht es aus, das Ergebnisliteral r zu invertieren.
Realisiert wird die Negation durch eine XOR-Klausel mit zwei Literalen (siehe Abbildung \ref{fig:gatter_not_cnf}). Diese wird nur dann zu $1$ evaluiert, wenn
a und r unterschiedlich sind, was genau der Negation entspricht. Falls XOR-Klauseln nicht unterstützt werden, lässt sich die Negation mit zwei Klauseln à zwei
Literalen realisieren, wie ebenfalls in der Abbildung dargestellt. Auch diese konjunktive Normalform deckt sich mit der Angabe in
"`Digitaltechnik - Eine praxisnahe Einführung"' \cite[164]{digitaltechnik}.
\begin{figure}[!h]
  \centering
  \begin{minipage}[l]{1cm}
    ~\\
    ~
  \end{minipage}
  \begin{minipage}[l]{2.5cm}
    \underline{Ohne XOR}\\
    $ (r \vee a) ~ \wedge $\\
    $ (\overline{r} \vee \overline{a}) $
  \end{minipage}
  \begin{minipage}[l]{2.5cm}
    \underline{Mit XOR}\\
    $ (r \veebar a) $\\
    ~
  \end{minipage}
  \caption{Gatter - NOT - Konjuktive Normalform}
  \label{fig:gatter_not_cnf}
\end{figure}
\section{Addierer}
\label{sec:knf:addierer}

Grundlage für die modulare 32-Bit Addition sind der Halbaddierer, der Volladdierer und der Mod-2 Addierer (siehe Abschnitt \ref{sec:grundlagen_add}).
Wie auch für die Gatter im Abschnitt vorher wird für den Halbaddierer eine boolsche Gleichungen erstellt (siehe Abbildung \ref{fig:halfadder_eqn}).
Diese entspricht genau den Gattern in Abbildung \ref{fig:halfadder}.
\begin{figure}[!h]
  \centering
  \begin{lstlisting}[]
  NAME = HalfAdder;
  INORDER = o_out s_out a_in b_in;
  OUTORDER = z;

  z = eq(s_out, xor(a_in, b_in)) & eq(o_out, a_in & b_in);
  \end{lstlisting}
  \caption{Halbaddierer - Gleichung}
  \label{fig:halfadder_eqn}
\end{figure}

Nach der Verwendung von eqntott und Espresso ergibt sich aus der boolenschen Gleichung die konjunktive Normalform in Abbildung \ref{fig:halfadder_cnf} (linke Seite).
eqntott und Espresso unterstützen jedoch keine XOR-Klauseln, weshalb ein manueller Versuch unternommen wird, diesen Vorteil zu nutzen. Das Ergebnis ist in Abbildung
\ref{fig:halfadder_cnf} (rechte Seite) dargestellt und besteht aus dem AND- und dem XOR-Gatter. Die Klauselmenge kann so von sechs auf vier Klauseln reduziert werden.
Diese Lösung bietet sich jedoch nur an, wenn die XOR-Klausel auch direkt verwendet werden kann. Wird sie in normale Klauseln umgewandelt, stehen im Ergebnis sieben
Klauseln. Das ist eine mehr als in der Lösung von Espresso.
\begin{figure}[!h]
  \centering
  \begin{minipage}[l]{5cm}
    ~~~~~~~~\underline{Ohne XOR}\\
    $ (\overline{s} \vee a \vee b) ~ \wedge $\\
    $ (\overline{o} \vee \overline{s}) ~ \wedge $\\
    $ (\overline{o} \vee b) ~ \wedge $\\
    $ (o \vee \overline{a} \vee \overline{b}) ~ \wedge $\\
    $ (s \vee a \vee \overline{b}) ~ \wedge $\\
    $ (s \vee \overline{a} \vee b) $
  \end{minipage}
  \begin{minipage}[l]{5cm}
    ~~~~~~~~\underline{Mit XOR}\\
    \underline{Übertrag - AND}\\
    $ (\overline{o} \vee a) ~ \wedge $\\
    $ (\overline{o} \vee b) ~ \wedge $\\
    $ (o \vee \overline{a} \vee \overline{b}) ~ \wedge $\\
    \underline{Summe - XOR}\\
    $ (\overline{s} \veebar a \veebar b) $
  \end{minipage}
  \caption{Halbaddierer - Konjuktive Normalform}
  \label{fig:halfadder_cnf}
\end{figure}

Im Volladdierer (siehe Abbildung \ref{fig:fulladder}) gibt es keine Gatter, die eine direkte Beziehung zwischen Eingangs- und Ausgangs-Literalen definieren.
Für die direkte Realisierung mit Hilfe der Gatter müssten 3 weitere Literale eingefügt werden. Für zwei AND- und ein OR-Gatter ergeben sich neun Klauseln (jeweils drei).
Bleiben noch die beiden XOR-Gatter die jeweils zu einer XOR-Klausel oder vier normalen Klauseln führen. Die Anzahl der Literale beträgt somit insgesamt 8 und die Anzahl der
Klauseln 17 (oder 9 + 2 XOR).

Bei der Verwendung von eqntott und Espresso zeigt sich, dass es kompaktere Lösungen gibt, die ohne zusätzliche Literale weniger Klauseln benötigen.
Die boolsche Gleichung für den Volladdierer ist in Abbildung \ref{fig:fulladder_qen} dargestellt. Wie auch bei dem Halbaddierer werden zwei mögliche
Lösungen generiert. Während in der ersten Lösung die vollständige boolsche Gleichung genutzt wird, fallen in der zweiten Lösung die blau markierten
Teile weg, um die XOR-Operationen manuell zu berücksichtigen.
\begin{figure}[!h]
  \centering
  \lstset{moredelim=**[is][\color{blue}]{@}{@}, moredelim=**[is][\bfseries]{§}{§}}
  \begin{lstlisting}[]
  NAME = FullAdder;
  INORDER = o_out §@s_out@§ a_in b_in c_in;
  OUTORDER = z;

  z = §@eq(s_out, xor(xor(a_in, b_in), c_in)) &@§ eq(o_out, (a_in & b_in) | (xor(a_in, b_in) & c_in));
  \end{lstlisting}
  \caption{Volladdierer - Gleichung}
  \label{fig:fulladder_qen}
\end{figure}

Es ergeben sich die konjunktiven Normalformen in Abbildung \ref{fig:fulladder_cnf}. Der eingehende Übertrag wird als c bezeichnet während der
berechnete Übertrag o genannt wird. Die Anzahl der Klauseln beläuft sich auf zehn, wobei durch die XOR-Unterstützung
eine Reduktion auf 7 Klauseln möglich ist. Die Umwandlung der XOR-Klausel in normale Klauseln würde zu einer Menge von 14 Klauseln führen.
\begin{figure}[!h]
  \centering
  \begin{minipage}[l]{5cm}
    ~~~~~~~~\underline{Ohne XOR}\\
    $ (s \vee \overline{a} \vee \overline{b} \vee \overline{c}) ~ \wedge $\\
    $ (\overline{s} \vee a \vee b \vee c) ~ \wedge $\\
    $ (o \vee \overline{a} \vee \overline{b}) ~ \wedge $\\
    $ (\overline{o} \vee a \vee b) ~ \wedge $\\
    $ (o \vee s \vee \overline{c}) ~ \wedge $\\
    $ (\overline{o} \vee \overline{s} \vee c) ~ \wedge $\\
    $ (\overline{s} \vee a \vee \overline{b} \vee \overline{c}) ~ \wedge $\\
    $ (\overline{s} \vee \overline{a} \vee b \vee \overline{c}) ~ \wedge $\\
    $ (s \vee a \vee \overline{b} \vee c) ~ \wedge $\\
    $ (s \vee \overline{a} \vee b \vee c) $
  \end{minipage}
  \begin{minipage}[l]{5cm}
    ~~~~~~~~\underline{Mit XOR}\\
    \underline{Übertrag}\\
    $ (o \vee \overline{a} \vee \overline{b}) ~ \wedge $\\
    $ (\overline{o} \vee a \vee b) ~ \wedge $\\
    $ (o \vee \overline{a} \vee \overline{c}) ~ \wedge $\\
    $ (o \vee \overline{b} \vee \overline{c}) ~ \wedge $\\
    $ (\overline{o} \vee a \vee c) ~ \wedge $\\
    $ (\overline{o} \vee b \vee c) ~ \wedge $\\
    \underline{Summe - XOR}\\
    $ (\overline{s} \veebar a \veebar b \veebar c) $\\
    ~
  \end{minipage}
  \caption{Volladdierer - Konjuktive Normalform}
  \label{fig:fulladder_cnf}
\end{figure}

Der Mod-2 Addierer braucht keiner weiteren Analyse unterzogen werden, da er ausschließlich aus XOR-Gattern besteht.
Die konjuktive Normalform ist in Abbildung \ref{fig:lastadder_cnf} dargestellt. Es werden entweder acht normale Klauseln
oder eine XOR-Klausel benötigt. Eine Lösung mit zwei einzelnen XOR-Gattern würde keine Vorteile bringen, da sie ein
zusätzliches Literal benötigt und die Klauselanzahl nicht verringert.
\begin{figure}[!h]
  \centering
  \begin{minipage}[l]{5cm}
    ~~~~~~~~\underline{Ohne XOR}\\
    $ (\overline{s} \vee a \vee b \vee c) $\\
    $ (\overline{s} \vee a \vee \overline{b} \vee \overline{c}) $\\
    $ (\overline{s} \vee \overline{a} \vee b \vee \overline{c}) $\\
    $ (\overline{s} \vee \overline{a} \vee \overline{b} \vee c) $\\
    $ (s \vee \overline{a} \vee \overline{b} \vee \overline{c}) $\\
    $ (s \vee \overline{a} \vee b \vee c) $\\
    $ (s \vee a \vee \overline{b} \vee c) $\\
    $ (s \vee a \vee b \vee \overline{c}) $
  \end{minipage}
  \begin{minipage}[l]{5cm}
    ~~~~~~~~\underline{Mit XOR}\\
    $ (\overline{s} \veebar a \veebar b \veebar c) $\\
    ~\\
    ~\\
    ~\\
    ~\\
    ~\\
    ~\\
    ~
  \end{minipage}
  \caption{Mod-2 Addierer - Konjuktive Normalform}
  \label{fig:lastadder_cnf}
\end{figure}

In Tabelle \ref{fig:add_literalclausecount} sind noch mal alle Ergebnisse aufgelistet. Die erste Zahl steht jeweils für die Anzahl der Literale
und die zweite Zahl für die Anzahl der Klauseln. In der letzten Spalte sind die Werte für einen modularen 32-Bit Addierer aufgelistet.
Die Anzahl der Klauseln ergibt sich aus einem Halbaddierer, 30 Volladdierern und einem Mod-2 Addierer. Die Anzahl der Literale wird zunächst
auf die gleiche Weise berechnet. Es müssen jedoch noch 31 Literale abgezogen werden, da die Literale für die Überträge doppelt berechnet wurden.
\begin{table}[!h]
  \centering
  \begin{tabular}{l|rrr|r}
    \hiderowcolors
                           & Halbaddierer & Volladdierer & Mod-2 Addierer &    Gesamt \\
    \hline
    Gatter                 &        4 - 7 &       8 - 17 &          4 - 8 & 217 - 525 \\
    Gatter mit XOR         &        4 - 4 &       8 - 11 &          4 - 1 & 217 - 335 \\
    eqntott \& Espresso    &        4 - 6 &       5 - 10 &          4 - 8 & 127 - 314 \\
    XOR                    &        4 - 4 &      5 - ~~7 &          4 - 1 & 127 - 215 \\
    XOR ohne Unterstützung &        4 - 7 &       5 - 14 &          4 - 8 & 127 - 435 \\
    \showrowcolors
  \end{tabular}
  \caption{Addierer - Literale und Klauseln}
  \label{fig:add_literalclausecount}
\end{table}

Es zeigt sich, dass die Verwendung von eqntott und Espresso, im Vergleich zur Verwendung der Gatter, zu einer kompakteren konjunktiven Normalform führt.
Durch die Verwendung von XOR-Klauseln, lässt sich die Klauselmenge weiter reduzieren. Dies sollte jedoch nicht die Basis für eine Nutzung ohne XOR-Unterstützung
sein, da in diesem Fall die Klauselmenge größer ist, als die Klauselmenge der von eqntott und Espresso berechneten Version. Es werden daher beide Versionen
implementiert, so dass sich je nach Anwendungsfall die bessere Version nutzen lässt.

Ein Anwendungsfall ohne XOR-Unterstützung ist die Generierung von Wahrheitstabellen. Die Komponenten des modularen 32-Bit Addierers wurden bisher einzeln
erzeugt und zusammengefügt. Die kompakte Repräsentation ermöglicht es, den 32-Bit Addierer vollständig in einer Wahrheitstabelle darzustellen.
Die Wahrheitstabelle hat dabei 127 Spalten mit 314 Zeilen. Diese Wahrheitstabelle dient wieder als Eingabe für Espresso, um weitere Optimierungen zu
ermöglichen, die sich aus dem Zusammenhang der einzelnen Komponenten ergeben könnten. Es stellt sich aber heraus, dass Espresso mit 127 Literalen
überfordert ist, und auch nach mehr als 24 Stunden zu keiner Lösung kommt.
\section{Addierer (Konstante)}

\TODO{erledigen}
\section{Choose (CH)}

\TODO{erledigen}
\section{Majority (MAJ)}

\TODO{erledigen}

espresso hat gute lösung\\
wahrheitstabelle\\
warum reichen die klauseln?
\section{Sigma-Familie (SIG)}
\label{sec:knf:sig}

Im Gegensatz zu den Funktionen in den vorhergehenden Abschnitten wird bei den Sigma-Funktionen ausschließlich das XOR-Gatter verwendet.
Während bisher mehrere Eingaben verrechnet wurden, wird bei den Sigma-Funktionen nur eine einzelne Eingabe verwendet, deren
einzelne Bits Verwendung in bis zu drei unterschiedlichen Ausgabebits finden. Da jedes Ausgabebit direkt durch ein XOR-Gatter aus den
Eingabebits berechnet wird, sind keine zusätzlichen Literale notwendig. Aus 32 Eingabebits und 32 Ausgabebits ergeben sich 64 Literale.

Bei Verwendung der XOR-Klauseln reicht eine Klausel für jedes Ausgabebit aus, unabhängig davon, ob das Ausgabebit aus zwei oder drei
Eingangsbits berechnet wird. Anders ist es bei fehlender XOR-Unterstützung. Ein XOR mit drei Eingängen benötigt acht Klauseln, während
ein XOR mit zwei Eingängen vier Klauseln benötigt. Bei den $ \Sigma $-Funktionen kommt dieser Unterschied nicht zum Tragen, weil generell
drei Eingabebits für jedes Ausgabebit herangezogen werden. Es ergeben sich 256 Klauseln. Anders verhält es sich bei den $ \sigma $-Funktionen.
Durch die Verschiebung nach rechts werden 3 bzw. 10 Eingangsbits nur zwei mal berücksichtigt. Die entstehende $0$ hat keinen Einfluss auf das
Ergebnis einer XOR-Operation und kann ignoriert werden. Es ergeben sich deshalb nur 244 bzw. 216 Klauseln. Eine Übersicht ist in Tabelle
\ref{fig:sigma_literalclausecount} dargestellt.
\begin{table}[!h]
  \centering
  \begin{tabular}{l|r|r|r|r}
    \hiderowcolors
                    & \multicolumn{4}{c}{\textbf{Funktion}} \\
    \cline{2-5}
    \textbf{Lösung} & $ \boldsymbol{\sigma_0} $ & $ \boldsymbol{\sigma_1} $ & $ \boldsymbol{\Sigma_0} $ & $ \boldsymbol{\Sigma_1} $ \\
    \hline
    \showrowcolors
    Gatter ohne XOR &     64 - 244 &     64 - 216 &     64 - 256 &     64 - 256 \\
    \hline
    Gatter mit XOR  &    64 - ~~32 &    64 - ~~32 &    64 - ~~32 &    64 - ~~32 \\
  \end{tabular}
  \caption{Sigma - Literale und Klauseln}
  \label{fig:sigma_literalclausecount}
\end{table}

Wie auch für den Addierer, führt der Versuch, die vier Funktionen als Wahrheitstabelle auszugeben und mit Espresso zu optimieren, zu keinem Ergebnis.
\section{Übergeordnete Module}

\TODO{erledigen}

Addierer:\\
BSIG0 / MAJ\\
BSIG1 / CH\\
SSIG0\\
SSIG1\\
Erweiterungsfunktion\\
Rundenfunktion\\
Rundenfunktion + Konstante\\
SHA256 als ganzes: wissen hinzufügen und mehrfache verwendung bei bitcoin\\
~\\
Rekonvergenzen bei ersten beiden finden.

\section{Vergleich mit anderen Implementierungen}

Für die vollständige Kompressionsfunktion werden bei diesem Vorgehen 49.832 Literale benötigt. Je nach Unterstützung von XOR-Klauseln
werden 255.600 bzw. 150.760 Klauseln benötigt, um die Kompressionsfunktion auf diesen Literalen zu definieren. Nicht mit eingerechnet
sind dabei Klauseln, die für eine Wertebelegung benötigt werden. Soll ein Bitcoin-Block berechnet werden, muss die Kompressionsfunktion
zwei Mal angewendet werden (siehe Abschnitt \ref{sec:bitcoin}). Das führt im Idealfall zu einer Verdopplung der Klauseln und Literale,
wobei 256 Literale wegfallen. Diese Literale stehen für das Ergebnis der ersten Anwendung der Kompressionsfunktion und dienen als Eingabe
für die zweite Anwendung der Kompressionsfunktion.

Jonathan Heusser hat \acr{cbmc} genutzt, um die konjunktive Normalform aus einem C-Programm zu generieren. Sein Fokus lag dabei auf der
Berechnung eines Bitcoin-Blocks. Da \acr{cbmc} keine XOR-Klauseln unterstützt, können ausschließlich normale Klauseln zum Vergleich
herangezogen werden. Aus seiner Datei satcoin.c \cite{jona:3} wurde eine konjunktive Normalform mit 132.615 Literalen und 648.233 Klauseln
generiert. Für die Berechnung einer einzelnen Kompressionsfunktion wird das C-Programm entsprechend angepasst (siehe Anhang \ref{chp:sha256code}).
Daraus ergeben sich 69.356 Literale und 347.128 Klauseln. Hier zeigt sich, dass \acr{cbmc} für die Berechnung eines Bitcoin-Blocks weniger
als das Doppelte einer Kompressionsfunktion notwendig ist. Das liegt daran, dass \acr{cbmc} die Operationen, deren Eingaben bekannt sind,
vor berechnet und nur die unbekannten Teile in die konjunktive Normalform überführt. Anzumerken ist auch, dass diese Klauselmengen bereits
Belegungen für konkret Werte enthalten, was jedoch zu vernachlässigen ist. 

Martin Maurer hat ein ähnliches Konzept wie das aus dieser Arbeit verwendet und die konjunktive Normalform selbst generiert. Neben der
Unterstützung für XOR-Klauseln erzeugt sein Programm \cite{capiman} zwei verschiedene Version der konjunktiven Normalform. Die erste nutzt
die Tseitin-Transformation auf Ebene der Gatter, während die zweite Espresso mit einbezieht um zusätzliche Literale in den Addierern zu vermeiden.
Die Zahlen zu dieser Lösung sind in Tabelle \ref{fig:sha256_literalclausecount} neben den beiden anderen Lösungen dargestellt. Wie auch bei der
Realisierung in dieser Arbeit lässt sich die Verdopplung der Literale und Klauseln erkennen, wenn ein Bitcoin-Block berechnet werden soll.
\begin{table}[!h]
  \centering
  \begin{tabular}{l|l|r|r|r}
    \hiderowcolors
    Problem                          & Realisierung & Literale & Klauseln & Klauseln (XOR) \\
    \hline
    \multirow{4}{2cm}{\glos{sha256}} & Diese Arbeit      &    49832 &   255600 ~~(~~5,13 / 3,58) &  150760 ~~(3,03 / 3,12) \\
                                     & CBMC              &    69356 &   347128 ~~(~~5,01 / 3,01) &                     --- \\
                                     & Maurer - Tseitin  &   130209 &   449929 ~~(~~3,46 / 2,77) &  261777 ~~(2,10 / 2,46) \\
                                     & Maurer - Espresso &    60161 &   665345 ~~( 11,06 / 5,60) &  590593 ~~(9,82 / 5,87) \\
    \hline
    \multirow{4}{2cm}{Bitcoin}       & Diese Arbeit      &    99408 &   511200 ~~(~~5,14 / 3,58) &  301520 ~~(3,03 / 3,12) \\
                                     & CBMC              &   132615 &   648233 ~~(~~4,89 / 3,01) &                     --- \\
                                     & Maurer - Tseitin  &   260673 &   901137 ~~(~~3,46 / 2,77) &  524321 ~~(2,01 / 2,46) \\
                                     & Maurer - Espresso &   120577 &  1331969 ~~( 11,05 / 5,60) & 1181953 ~~(9,80 / 5,87) \\
    \showrowcolors
  \end{tabular}
  \caption{Vergleich der Anzahl von Literalen und Klauseln}
  \label{fig:sha256_literalclausecount}
\end{table}

Der Vergleich zeigt, dass das Vorgehen aus dieser Arbeit zu einer kompakteren konjunktiven Normalform führt.
Sowohl die Anzahl der Literale als auch der Klauseln sind im Bezug zu den anderen Realisierungen minimal.

Auffällig ist, dass Martin Maurer mit seiner Espresso Variante ebenfalls wenig Literale benötigt, jedoch die
Anzahl der Klauseln am höchsten von allen Realisierungen ist. Auf seiner GitHub-Seite \cite{capiman} schreibt
er: "`In my tests with CMS 3.x the version with TSEITIN ADDERS were faster than version with ESPRESSO."'.
Das legt die Vermutung nahe, dass die Anzahl der Klauseln ein wesentlicher Faktor für die Geschwindigkeit ist.

Aufgeführt sind in Tabelle \ref{fig:sha256_literalclausecount} für jede Klauselmenge in Klammern zwei weitere Werte.
Der erste Wert beschreibt das Verhältnis von Klauseln zu Literalen. In dieser Arbeit werden ohne XOR-Unterstützung
bspw. 5,14 Klauseln pro Literal benötigt. Der zweite Wert beschreibt die durchschnittliche Klausellänge.
Im genannten Beispiel bezieht sich eine Klausel durchschnittlich auf 3,58 Literale.

Bei der Betrachtung dieser Werte sticht die Tseitin-Variante von Martin Maurer hervor. Die Gatter AND und OR mit
zwei Eingängen lassen sich jeweils durch drei Klauseln beschreiben. Zwei Klauseln beziehen sich dabei auf jeweils
zwei Literale und eine Klausel auf drei Literale. Ein XOR-Gatter lässt sich mit vier Klauseln à drei Literale beschreiben.
Die durchschnittliche Klausellänge liegt damit zwischen zwei und drei und die durchschnittliche Klauselmenge pro Literal
zwischen drei und vier. Durch eine XOR-Unterstützung lässt sich die durchschnittliche Klauselmenge auf eins bis drei
reduzieren. Diese Werte passen zu den ermittelten Werten in der Tabelle. Im Gegensatz zu seiner Espresso-Variante ist
die Realisierung der Tseitin-Variante gelungen.