\section{Sigma-Familie (SIG)}
\label{sec:knf:sig}

Im Gegensatz zu den Funktionen in den vorhergehenden Abschnitten wird bei den Sigma-Funktionen ausschließlich das XOR-Gatter verwendet.
Während bisher mehrere Eingaben verrechnet wurden, wird bei den Sigma-Funktionen nur eine einzelne Eingabe verwendet, deren
einzelne Bits Verwendung in bis zu drei unterschiedlichen Ausgabebits finden. Da jedes Ausgabebit direkt durch ein XOR-Gatter aus den
Eingabebits berechnet wird, sind keine zusätzlichen Literale notwendig. Aus 32 Eingabebits und 32 Ausgabebits ergeben sich 64 Literale.

Bei Verwendung der XOR-Klauseln reicht eine Klausel für jedes Ausgabebit aus, unabhängig davon, ob das Ausgabebit aus zwei oder drei
Eingangsbits berechnet wird. Anders ist es bei fehlender XOR-Unterstützung. Ein XOR mit drei Eingängen benötigt acht Klauseln, während
ein XOR mit zwei Eingängen vier Klauseln benötigt. Bei den $ \Sigma $-Funktionen kommt dieser Unterschied nicht zum Tragen, weil generell
drei Eingabebits für jedes Ausgabebit herangezogen werden. Es ergeben sich 256 Klauseln. Anders verhält es sich bei den $ \sigma $-Funktionen.
Durch die Verschiebung nach rechts werden 3 bzw. 10 Eingangsbits nur zwei mal berücksichtigt. Die entstehende $0$ hat keinen Einfluss auf das
Ergebnis einer XOR-Operation und kann ignoriert werden. Es ergeben sich deshalb nur 244 bzw. 216 Klauseln. Eine Übersicht ist in Tabelle
\ref{fig:sigma_literalclausecount} dargestellt.
\begin{table}[!h]
  \centering
  \begin{tabular}{l|r|r|r|r}
    \hiderowcolors
                    & \multicolumn{4}{c}{\textbf{Funktion}} \\
    \cline{2-5}
    \textbf{Lösung} & $ \boldsymbol{\sigma_0} $ & $ \boldsymbol{\sigma_1} $ & $ \boldsymbol{\Sigma_0} $ & $ \boldsymbol{\Sigma_1} $ \\
    \hline
    \showrowcolors
    Gatter ohne XOR &     64 - 244 &     64 - 216 &     64 - 256 &     64 - 256 \\
    \hline
    Gatter mit XOR  &    64 - ~~32 &    64 - ~~32 &    64 - ~~32 &    64 - ~~32 \\
  \end{tabular}
  \caption{Sigma - Literale und Klauseln}
  \label{fig:sigma_literalclausecount}
\end{table}

Wie auch für den Addierer, führt der Versuch, die vier Funktionen als Wahrheitstabelle auszugeben und mit Espresso zu optimieren, zu keinem Ergebnis.