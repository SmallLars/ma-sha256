\section{Choose (CH)}
\label{sec:knf:ch}

Die Choose-Funktion kann durch zwei AND-Gatter und ein XOR-Gatter realisiert werden. Als Alternative für das XOR-Gatter wird auch ein OR-Gatter berücksichtigt.
Bei Verwendung der Gatter müssen 2 zusätzliche Literale eingefügt werden, die das Ergebnis der beiden AND-Gatter enthalten, so dass insgesamt 6 Literale benötigt werden.
Für die AND-Gatter werden jeweils drei Klauseln benötigt. Das XOR-Gatter benötigt eine XOR-Klausel oder vier normale Klauseln, während das OR-Gatter drei Klauseln benötigt.
Je nach Realisierung ergeben sich damit zehn, neun oder sieben Klauseln.

Nach den Erfolgen in den beiden vorhergehenden Abschnitten wird für die Choose-Funktion ebenfalls eine Boolesche Gleichung erstellt (siehe Abbildung \ref{fig:choose_eqn}).
Im Gegensatz zu den Addierern ist das Ergebnis des XOR-Gatters nicht direkt von Eingaben abhängig, sondern von anderen Gattern. Damit lässt sich das XOR-Gatter nicht
ausgliedern und separat betrachten, so dass nur die vollständige Boolesche Gleichung genutzt wird.
\begin{figure}[!h]
  \centering
  \begin{lstlisting}[]
  NAME = CH;
  INORDER = r_out a_in b_in c_in;
  OUTORDER = z;

  z = eq(r_out, xor(a_in & b_in, !a_in & c_in));
  \end{lstlisting}
  \caption{Choose - Gleichung}
  \label{fig:choose_eqn}
\end{figure}

Aus der Berechnung von eqntott und Espresso ergibt sich die konjunktive Normalform in Abbildung \ref{fig:choose_cnf}.
Vier Klauseln reichen aus, um die Choose-Funktion zu beschreiben.
\begin{figure}[!h]
  \centering
  \begin{minipage}[l]{2cm}
    $ (r \vee \overline{a} \vee \overline{b}) ~ \wedge $\\
    $ (\overline{r} \vee \overline{a} \vee b) ~ \wedge $\\
    $ (r \vee a \vee \overline{c}) ~ \wedge $\\
    $ (\overline{r} \vee a \vee c) $
  \end{minipage}
  \caption{Choose - Konjuktive Normalform}
  \label{fig:choose_cnf}
\end{figure}

In Tabelle \ref{fig:choose_literalclausecount} sind die unterschiedlichen Lösungen gegenübergestellt.
Die Choose-Funktion wird bitweise auf einer 32 Bit breiten Binärzahl angewendet. Jedes Bit der Eingabe wird dabei nur einmal genutzt.
Die einzelnen Berechnungen sind somit unabhängig, wodurch sich die Gesamtanzahl der Literale und Klauseln aus der Multiplikation mit 32 ergibt.
\begin{table}[!h]
  \centering
  \begin{tabular}{l|r|r}
    \hiderowcolors
    \textbf{Lösung}        & \textbf{Choose} & \textbf{Gesamt} \\
    \hline
    \showrowcolors
    Gatter                 &  6 -  10 & 192 - 320 \\
    Gatter mit OR          &  6 - ~~9 & 192 - 288 \\
    Gatter mit XOR         &  6 - ~~7 & 192 - 224 \\
    eqntott \& Espresso    &  4 - ~~4 & 128 - 128 \\
  \end{tabular}
  \caption{Choose - Literale und Klauseln}
  \label{fig:choose_literalclausecount}
\end{table}

Das Ergebnis zeigt, dass es nicht notwendig ist, unterschiedliche Lösungen bereitzustellen. Egal ob eine XOR-Unterstützung gegeben ist oder nicht,
ist die von eqntott und Espresso ermittelte Lösung optimal und wird implementiert.