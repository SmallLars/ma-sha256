\section{Gatter}
\label{sec:knf:gatter}

Grundlage für die Tseitin-Transformation (siehe Abschnitt \ref{sec:knf}) sollen in dieser Arbeit die Operatoren AND, OR und XOR bilden. Verwendet werden sie jedoch
nur, falls sie in einem Modul eine direkte Beziehung zwischen Eingangs- und Ausgang-Literalen definieren, da sonst zusätzliche Literale eingefügt werden müssen.
Die Boolesche Gleichungen für die drei Operatoren sind in Abbildung \ref{fig:gatter_equations} dargestellt. a und b werden als Literale für die Eingänge verwendet
und r steht für den Ausgang (das Resultat).

\begin{figure}[!h]
  \centering
  \begin{minipage}[c]{4.85cm}
    \begin{lstlisting}[]
  NAME = AND;
  INORDER = r_out a_in b_in;
  OUTORDER = z;

  z = eq(r_out, (a_in & b_in));
    \end{lstlisting}
  \end{minipage}
  \begin{minipage}[c]{4.85cm}
    \begin{lstlisting}[]
  NAME = OR;
  INORDER = r_out a_in b_in;
  OUTORDER = z;

  z = eq(r_out, (a_in | b_in));
    \end{lstlisting}
  \end{minipage}
  \begin{minipage}[c]{5.1cm}
    \begin{lstlisting}[]
  NAME = XOR;
  INORDER = r_out a_in b_in;
  OUTORDER = z;

  z = eq(r_out, xor(a_in, b_in));
    \end{lstlisting}
  \end{minipage}
  \caption{Gatter - Gleichungen}
  \label{fig:gatter_equations}
\end{figure}

Nach der Verwendung von eqntott und Espresso ergeben sich aus den Booleschen Gleichungen die konjunktiven Normalformen in Abbildung \ref{fig:gatter_cnf}.
Diese decken sich mit den Angaben aus "`Digitaltechnik - Eine praxisnahe Einführung"' \cite[164]{digitaltechnik}. Für die Negation der Operatoren reicht es aus, das Ergebnisliteral r zu invertieren.

\begin{figure}[!h]
  \centering
  \begin{minipage}[l]{4.65cm}
    \underline{AND}\\
    $ (\overline{r} \vee a) ~ \wedge $\\
    $ (\overline{r} \vee b) ~ \wedge $\\
    $ (r \vee \overline{a} \vee \overline{b}) $\\
    ~
  \end{minipage}
  \begin{minipage}[l]{4.65cm}
    \underline{OR}\\
    $ (r \vee \overline{a}) ~ \wedge $\\
    $ (r \vee \overline{b}) ~ \wedge $\\
    $ (\overline{r} \vee a \vee b) $\\
    ~
  \end{minipage}
  \begin{minipage}[l]{4.3cm}
    \underline{XOR}\\
    $ (\overline{r} \vee \overline{a} \vee \overline{b}) ~ \wedge $\\
    $ (r \vee a \vee \overline{b}) ~ \wedge $\\
    $ (r \vee \overline{a} \vee b) ~ \wedge $\\
    $ (\overline{r} \vee a \vee b) $
  \end{minipage}
  \caption{Gatter - Konjunktive Normalformen}
  \label{fig:gatter_cnf}
\end{figure}

Während es bei den Operatoren AND und OR gelungen ist, Don't-Care Literale zu identifizieren und für eine Vereinfachung zu nutzen, ist dies bei dem XOR-Operator
nicht möglich. Jede einzelne Änderung eines beliebigen Eingangsliterals führt zu einer Änderung des Ausgangssignals. Genau die Hälfte der möglichen Belegungen
ist somit nicht erfüllbar und muss in die konjunktive Normalform mit aufgenommen werden. Allgemein gilt, dass bei einem XOR mit n Eingängen $ 2^{n} $ Klauseln
mit jeweils $ (n + 1) $ Literalen notwendig sind. Die Klauselmenge wächst exponentiell im Bezug zur Anzahl der Eingänge.

CryptoMiniSat (siehe Abschnitt \ref{sec:satsolver}) ist in der Lage, neben normalen Klauseln (Disjunktion) auch Klauseln zu berücksichtigen, deren Literale
mit dem XOR-Operator verknüpft sind. Genau wie normale Klauseln müssen sie für die Erfüllbarkeit zu $1$ evaluiert werden. Damit lässt sich das exponentielle
Wachstum der Klauselmenge umgehen. Für ein XOR mit n Eingängen ist nur noch eine Klausel mit $ (n + 1) $ Literalen notwendig. Abbildung \ref{fig:gatter_cnf_xor}
zeigt die Klausel für den XOR-Operator mit zwei und drei Eingängen.
\begin{figure}[!h]
  \centering
  \begin{minipage}[l]{1cm}
    ~\\
    ~
  \end{minipage}
  \begin{minipage}[l]{2.5cm}
    Zwei Eingänge:\\
    Drei Eingänge:
  \end{minipage}
  \begin{minipage}[l]{3cm}
    $ (\overline{r} \veebar a \veebar b) $\\
    $ (\overline{r} \veebar a \veebar b \veebar c) $
  \end{minipage}
  \caption{Gatter - XOR}
  \label{fig:gatter_cnf_xor}
\end{figure}

Abschließend wird noch ein Operator für die Negation (NOT) benötigt. Um daraus einen Gleichheits-Operator (EQ) zu machen reicht es aus, das Ergebnisliteral r zu invertieren.
Realisiert wird die Negation durch eine XOR-Klausel mit zwei Literalen (siehe Abbildung \ref{fig:gatter_not_cnf}). Diese wird nur dann zu $1$ evaluiert, wenn
a und r unterschiedlich sind, was genau der Negation entspricht. Falls XOR-Klauseln nicht unterstützt werden, lässt sich die Negation mit zwei Klauseln à zwei
Literalen realisieren, wie ebenfalls in der Abbildung dargestellt. Auch diese konjunktive Normalform deckt sich mit der Angabe in
"`Digitaltechnik - Eine praxisnahe Einführung"' \cite[164]{digitaltechnik}.
\begin{figure}[!h]
  \centering
  \begin{minipage}[l]{1cm}
    ~\\
    ~
  \end{minipage}
  \begin{minipage}[l]{2.5cm}
    \underline{Ohne XOR}\\
    $ (r \vee a) ~ \wedge $\\
    $ (\overline{r} \vee \overline{a}) $
  \end{minipage}
  \begin{minipage}[l]{2.5cm}
    \underline{Mit XOR}\\
    $ (r \veebar a) $\\
    ~
  \end{minipage}
  \caption{Gatter - NOT - Konjuktive Normalform}
  \label{fig:gatter_not_cnf}
\end{figure}