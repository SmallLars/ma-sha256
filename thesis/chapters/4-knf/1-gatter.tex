\section{Gatter}
\label{sec:knf:gatter}

\begin{figure}[!h]
  \centering
  \begin{minipage}[c]{4.85cm}
    \begin{lstlisting}[]
  NAME = AND;
  INORDER = r_out a_in b_in;
  OUTORDER = z;

  z = eq(r_out, (a_in & b_in));
    \end{lstlisting}
  \end{minipage}
  \begin{minipage}[c]{4.85cm}
    \begin{lstlisting}[]
  NAME = OR;
  INORDER = r_out a_in b_in;
  OUTORDER = z;

  z = eq(r_out, (a_in | b_in));
    \end{lstlisting}
  \end{minipage}
  \begin{minipage}[c]{5.1cm}
    \begin{lstlisting}[]
  NAME = XOR;
  INORDER = r_out a_in b_in;
  OUTORDER = z;

  z = eq(r_out, xor(a_in, b_in));
    \end{lstlisting}
  \end{minipage}
  \caption{Gattergleichungen}
  \label{fig:gatter_equations}
\end{figure}

hmm\TODO{erledigen}

and\\
or\\
xor\\
negation durch invertieren des ergebnisliterals
~\\
normales xor, negation eines literals reicht.\\
umwandlung zeigen von a = b xor x zu 1/0 = a xor b xor c\\
~\\
tseiting\\
reproduktion mit espresso möglich
~\\
beachte: anzahl der literale\\
darauf bezogen geht teilweise besser\\
falls ohne zusatzliterale: tseitin ist toll\\