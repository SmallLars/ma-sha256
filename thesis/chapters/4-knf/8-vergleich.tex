\section{Vergleich mit anderen Implementierungen}

Für die vollständige Kompressionsfunktion werden bei diesem Vorgehen 49.832 Literale benötigt. Je nach Unterstützung von XOR-Klauseln
werden 255.600 bzw. 150.760 Klauseln benötigt, um die Kompressionsfunktion auf diesen Literalen zu definieren. Nicht mit eingerechnet
sind dabei Klauseln, die für eine Wertebelegung benötigt werden. Soll ein Bitcoin-Block berechnet werden, muss die Kompressionsfunktion
zwei Mal angewendet werden (siehe Abschnitt \ref{sec:bitcoin}). Das führt im Idealfall zu einer Verdopplung der Klauseln und Literale,
wobei 256 Literale wegfallen. Diese Literale stehen für das Ergebnis der ersten Anwendung der Kompressionsfunktion und dienen als Eingabe
für die zweite Anwendung der Kompressionsfunktion.

Jonathan Heusser hat \acr{cbmc} genutzt, um die konjunktive Normalform aus einem C-Programm zu generieren. Sein Fokus lag dabei auf der
Berechnung eines Bitcoin-Blocks. Da \acr{cbmc} keine XOR-Klauseln unterstützt, können ausschließlich normale Klauseln zum Vergleich
herangezogen werden. Aus seiner Datei satcoin.c \cite{jona:3} wurde eine konjunktive Normalform mit 132.615 Literalen und 648.233 Klauseln
generiert. Für die Berechnung einer einzelnen Kompressionsfunktion wird das C-Programm entsprechend angepasst (siehe Anhang \ref{chp:sha256code}).
Daraus ergeben sich 69.356 Literale und 347.128 Klauseln. Hier zeigt sich, dass \acr{cbmc} für die Berechnung eines Bitcoin-Blocks weniger
als das Doppelte einer Kompressionsfunktion notwendig ist. Das liegt daran, dass \acr{cbmc} die Operationen, deren Eingaben bekannt sind,
vor berechnet und nur die unbekannten Teile in die konjunktive Normalform überführt. Anzumerken ist auch, dass diese Klauselmengen bereits
Belegungen für konkret Werte enthalten, was jedoch zu vernachlässigen ist. 

Martin Maurer hat ein ähnliches Konzept wie das aus dieser Arbeit verwendet und die konjunktive Normalform selbst generiert. Neben der
Unterstützung für XOR-Klauseln erzeugt sein Programm \cite{capiman} zwei verschiedene Version der konjunktiven Normalform. Die erste nutzt
die Tseitin-Transformation auf Ebene der Gatter, während die zweite Espresso mit einbezieht, um zusätzliche Literale in den Addierern zu vermeiden.
Die Zahlen zu dieser Lösung sind in Tabelle \ref{fig:sha256_literalclausecount} neben den beiden anderen Lösungen dargestellt. Wie auch bei der
Realisierung in dieser Arbeit lässt sich die Verdopplung der Literale und Klauseln erkennen, wenn ein Bitcoin-Block berechnet werden soll.
\begin{table}[!h]
  \centering
  \begin{tabular}{l|l|r|r|r}
    \hiderowcolors
    \textbf{Problem}                 & \textbf{Realisierung} & \textbf{Literale} & \textbf{Klauseln} & \textbf{Klauseln (XOR)} \\
    \hline
    \showrowcolors
                                     & Diese Arbeit      &    49832 &   255600 ~~(~~5,13 / 3,58) &  150760 ~~(3,03 / 3,12) \\
                                     & \acr{cbmc}        &    69356 &   347128 ~~(~~5,01 / 3,01) &                     --- \\
    \glos{sha256}                    & Maurer - Tseitin  &   130209 &   449929 ~~(~~3,46 / 2,77) &  261777 ~~(2,10 / 2,46) \\
                                     & Maurer - Espresso &    60161 &    665345 ~~(11,06 / 5,60) &  590593 ~~(9,82 / 5,87) \\
    \hline
                                     & Diese Arbeit      &    99408 &   511200 ~~(~~5,14 / 3,58) &  301520 ~~(3,03 / 3,12) \\
                                     & \acr{cbmc}        &   132615 &   648233 ~~(~~4,89 / 3,01) &                     --- \\
    Bitcoin                          & Maurer - Tseitin  &   260673 &   901137 ~~(~~3,46 / 2,77) &  524321 ~~(2,01 / 2,46) \\
                                     & Maurer - Espresso &   120577 &   1331969 ~~(11,05 / 5,60) & 1181953 ~~(9,80 / 5,87) \\
  \end{tabular}
  \caption{Vergleich der Anzahl von Literalen und Klauseln}
  \label{fig:sha256_literalclausecount}
\end{table}

Der Vergleich zeigt, dass das Vorgehen aus dieser Arbeit zu einer kompakteren konjunktiven Normalform führt.
Sowohl die Anzahl der Literale als auch der Klauseln sind im Bezug zu den anderen Realisierungen minimal.

Auffällig ist, dass Martin Maurer mit seiner Espresso Variante ebenfalls wenig Literale benötigt, jedoch die
Anzahl der Klauseln am höchsten von allen Realisierungen ist. Auf seiner GitHub-Seite \cite{capiman} schreibt
er: "`In my tests with CMS 3.x the version with TSEITIN ADDERS were faster than version with ESPRESSO."'.
Das legt die Vermutung nahe, dass die Anzahl der Klauseln ein wesentlicher Faktor für die Geschwindigkeit ist.

Aufgeführt sind in Tabelle \ref{fig:sha256_literalclausecount} für jede Klauselmenge in Klammern zwei weitere Werte.
Der erste Wert beschreibt das Verhältnis von Klauseln zu Literalen. In dieser Arbeit werden ohne XOR-Unterstützung
bspw. 5,14 Klauseln pro Literal benötigt. Der zweite Wert beschreibt die durchschnittliche Klausellänge.
Im genannten Beispiel bezieht sich eine Klausel durchschnittlich auf 3,58 Literale.

Bei der Betrachtung dieser Werte sticht die Tseitin-Variante von Martin Maurer hervor. Die Gatter AND und OR mit
zwei Eingängen lassen sich jeweils durch drei Klauseln beschreiben. Zwei Klauseln beziehen sich dabei auf jeweils
zwei Literale und eine Klausel auf drei Literale. Ein XOR-Gatter lässt sich mit vier Klauseln à drei Literale beschreiben.
Die durchschnittliche Klausellänge liegt damit zwischen zwei und drei und die durchschnittliche Klauselmenge pro Literal
zwischen drei und vier. Durch eine XOR-Unterstützung lässt sich die durchschnittliche Klauselmenge auf eins bis drei
reduzieren. Diese Werte passen zu den ermittelten Werten in der Tabelle. Im Gegensatz zu seiner Espresso-Variante ist
die Realisierung der Tseitin-Variante gelungen.