\section{Bedeutung für das Bitcoin-Mining}

Jonathan Heusser führt in seinem Artikel \cite{jona:1} zwei Versuche mit unterschiedlichen SAT-Solvern durch.
Diese ließen sich für einen direkten Vergleich leider nicht reproduzieren, weil die vom ihm genutzte konjunktive
Normalform unterschiedlich zu der ist, die sich mit \acr{cbmc} aus seinem C-Programmcode erstellen lässt.
Außerdem gibt er in seinen Versuchen eine feste Anzahl führender Nullen für den \glos{hash} vor und variiert die
Anzahl der Stellen der Nonce, die noch berechnet werden müssen, um die Schwierigkeit zu verändern. Das eignet
sich zwar für den Vergleich unterschiedlicher SAT-Solver, ermöglicht jedoch keine direkte Aussage dazu, ob
sich dieses Vorgehen dafür eignet aktuelle Bitcoin-Blöcke zu lösen.

In der Praxis ist die Nonce vollständig unbekannt und die Schwierigkeit wird dadurch angepasst, dass die Anzahl
der geforderten führenden Nullen sich ändert. Ziel von Bitcoin ist es, die Schwierigkeit immer so anzupassen, dass im Mittel
alle zehn Minuten ein Block gelöst wird. Damit stellt sich die Frage, bei wie vielen führenden Nullen der SAT-Solver
innerhalb von 10 Minuten zu einer Lösung kommt. Tabelle \ref{fig:bitcoinzeros} zeigt die Ergebnisse der mit CryptoMiniSat
durchgeführten Versuche.

\begin{table}[!h]
  \centering
  \begin{tabular}{rr|rr}
    Geforderte Nullen & Dauer in Minuten & Erhaltene Nullen \\
    \hline
     9 &  0:43 &  9 \\
    10 &  1:55 & 10 \\
    11 &  8:32 & 11 \\
    12 &  5:29 & 14 \\
    13 & 10:39 & 16
  \end{tabular}
  \caption{Erworbene Klauseln in der Kompressionsfunktion nach Bereinigung}
  \label{fig:bitcoinzeros}
\end{table}

Für jede Anzahl der geforderten Nullen wurden jeweils fünf Versuche durchgeführt, die durch eine Zeile in Tabelle \ref{fig:bitcoinzeros}
repräsentiert werden. Es hat sich gezeigt, dass CryptoMiniSat innerhalb jeder Versuchsreihe jeweils die selbe Nonce ermittelt hat
und die Dauer nur wenig variierte. Die angegebene Dauer ist der arithmetische Mittelwert. Die letzte Spalte enthält die Anzahl der
führenden Nullen, die tatsächlich erreicht wurden. Da ein SAT-Solver die nicht festgelegten Stellen frei belegen kann, ist es möglich,
dass das Ergebnis mehr führende Nullen hat als gefordert. Dies ist jedoch zufällig und nur zur Vollständigkeit mit aufgeführt.
In annähernd zehn Minuten ist es somit möglich, eine Nonce für einen \glos{hash} mit bis zu 13 führenden Nullen zu ermitteln.

Da die aktuellen Blöcke einen \glos{hash} mit mindestens 68 führenden Nullen erfordern, zeigt dieser Versuch, dass SAT-Solving sich
derzeit nicht für den Einsatz in Bitcoin-Minern eignet. Betrachtet werden muss auch, dass in der Testumgebung 720.000 \glospl{hash} pro Sekunde
berechnet werden können. Damit kann alleine durch Probieren in zehn Minuten eine gültige Nonce für einen \glos{hash} mit 27 führenden
Nullen ermittelt werden.