\chapter{Fazit}
\label{chp:fazit}

\TODO{erledigen}

erfolgreich war die kompakte cnf\\
deutlich besser als maurer und cbmc\\
framework hat sich bewährt, auch für des kompakte knf erzeugt\\
link, jens erwähnen \cite{trillmann}, kommt auf 48 von 64 bit\\

klauseln auf modulebene am schnellsten\\
kombination aller möglichen minterme\\
ist aber ein problem, weil endliche menge, bringt einen nicht weiter\\
vielleicht in diesem fall besonders gut, weil lösung hauptsächlich durch probieren erfolgt\\
sha256 lässt erwartungsgemäß wenig muster erkennen\\
sat-solver findet immer direkt eine klausel zum propagieren und braucht nicht über andere gehen\\

nutzen von zusätzlichem wissen eher fraglich, trotzdem ein interssanter weg,\\
weil durch weitere zwischenergebnisse eventuell andere zusammenhänge erkannt werden können\\
erstaunlich ist dabei der unterschied zwischen mit und ohne xor\\

vergleichsweise wenig distanzklauseln haben zwar 20\% verbesserung erzielt, weg hat sich bewährt.\\
in dieser arbeit wurde graph von modulen erstellt. allgemein graph möglich mit literalen als knoten.\\
kanten werden zu allen literalen gezogen die mit dem einen literal zusammen in klauseln auftreten\\
problem ist die größe des graphen. wird sehr schnell groß. hatte in dieser arbeit ca 1000 knoten. damit lässt sich effizient arbeiten\\

es werden nach wie vor weitere distanzklauseln gefunden. identifikation einfach. lässt sich automatisieren.\\
die verallgemeinerung erfolgte mit einigen hilfstools in dieser arbeit von hand. kann zukünftig automatisiert werden\\
erfordert aber auch einen weiteren sat thread, der zur laufzeit die gültigkeit überprüft\\
dokumentation interessant, einfach gültige hinzufügen ist intransparent, zusammenhang geht flöten, welche klauseln durch verallgemeinerung entstanden sind / zusammen gehören\\

bei verallgemeinerung zu beachten: viele der distanzklauseln basieren auf carry bits. carry ripple addierer deshalb interessant.\\

ergebnis im vergleich zu stumpfen probieren ineffizient\\
kann vielleicht als basis dienen\\

generell interessant als open source projekt\\
web-projekt: gemeinsam zusätzliches wissen als klauseln sammeln\\
kann problemlos kombiniert werden\\