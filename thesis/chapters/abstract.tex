\section*{\abstractname}

Ziel dieser Arbeit ist es, mit Hilfe von CryptoMiniSat Wissen über den Hash-Algorithmus \glos{sha256}
zu erwerben um dieses Wissen in weitere Lösungsversuche einzubringen und damit den Lösungsprozess zu
beschleunigen. Dafür wird eine konjunktive Normalform erstellt und für die Analyse mit CryptoMiniSat genutzt.
Die Analyse erfolgt iterativ. Nach einem Lösungsversuch werden gelernte Klauseln extrahiert und den
betreffenden Bereichen von \glos{sha256} zugeordnet. Diese Klauseln werden in weitere Lösungsversuche direkt
mit eingebunden. Es zeigt sich, dass dadurch der Lösungsprozess beschleunigt werden kann. Ein praktisch
relevanter Angriff ist damit jedoch nicht möglich. Auch die Berechnung eines Bitcoin-Blocks lässt sich
damit nicht ausreichend schnell durchführen.