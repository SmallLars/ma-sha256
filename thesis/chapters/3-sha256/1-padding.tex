\section{Padding}
\label{sec:sha256:padding}

Um die Eingabe auf ein Vielfaches von 512 Bit zu erweitern, wird das Padding benötigt. Dabei wird ein einzelnes Bit mit dem Wert "`$1$"' an die Eingabe der Länge L gehängt.
Dann folgen K "`$0$"'en und abschließend ein 64 Bit Block der die Länge L in binärer Repräsentation enthält. K ergibt sich dabei durch die Formel ( L + K ) mod 512 = 447.
Hier zeigt sich, dass bei einem K von 0 die Eingabe 447 Bit lang werden kann, bevor die Kompressionsfunktion weiteres Mal ausgeführt werden muss, da ein zweiter 512 Bit Block
erzeugt wird.

\begin{figure}[!h]
  \centering
  \includegraphics[scale=0.4]{images/sha256padding}
  \caption{Schematische Darstellung des Paddings}
  \label{fig:sha256padding}
\end{figure}