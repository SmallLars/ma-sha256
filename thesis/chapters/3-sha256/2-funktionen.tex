\section{Funktionen}
\label{sec:sha256:funktionen}

Sowohl bei der Erweiterung der Eingabe als auch bei der Rundenfunktion werden einige Funktionen verwendet, die im folgendem Erläutert werden.
Die Funktion "`Small Sigma"' wird bei der Erweiterung der Eingabe verwendet, während die Funktionen "`Choose"', "`Majority"' und "`Big Sigma"' in der Rundenfunktion
verwendet werden.

\subsection{Small Sigma (SSIG)}
\begin{eqnarray}
\sigma_0(x) = \text{ROTR}^{7}(x)~\oplus~\text{ROTR}^{18}(x)~\oplus~\text{SHR}^{3}(x)\\
\sigma_1(x) = \text{ROTR}^{17}(x)~\oplus~\text{ROTR}^{19}(x)~\oplus~\text{SHR}^{10}(x)
\end{eqnarray}

\subsection{Choose (CH)}
\begin{equation}
\text{CH}( x, y, z) = (x~\wedge~y)~\oplus~( \neg~x~\wedge~z)
\end{equation}

\subsection{Majority (MAJ)}
\begin{equation}
\text{MAJ}( x, y, z) = (x~\wedge~y)~\oplus~(x~\wedge~z)~\oplus~(y~\wedge~z)
\end{equation}

\subsection{Big Sigma (BSIG)}
\begin{eqnarray}
\Sigma_0(x) = \text{ROTR}^{2}(x)~\oplus~\text{ROTR}^{13}(x)~\oplus~\text{ROTR}^{22}(x)\\
\Sigma_1(x) = \text{ROTR}^{6}(x)~\oplus~\text{ROTR}^{11}(x)~\oplus~\text{ROTR}^{25}(x)
\end{eqnarray}