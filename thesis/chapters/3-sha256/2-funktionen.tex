\section{Funktionen}
\label{sec:sha256:funktionen}

Die Funktion, die in SHA-$256$ am häufigsten verwendet wird, ist die modulare 32-Bit Addition (siehe Abschnitt \ref{sec:grundlagen_add}).
Darüber hinaus werden sowohl bei der Erweiterung der Eingabe als auch bei der Rundenfunktion einige Funktionen verwendet, die im folgendem Erläutert werden.
Die Funktion "`Small Sigma"' wird bei der Erweiterung der Eingabe verwendet, während die Funktionen "`Choose"', "`Majority"' und "`Big Sigma"' in der Rundenfunktion
verwendet werden. Alle genannten Funktionen operieren auf einer Datenwortbreite von 32 Bit.

%Während die Aufgabe der Funktionen "`Choose"' und "`Majority"' die Konfusion ist, ist die Aufgabe der Sigma-Funktionen die Diffusion.
%Bei der Konfusion geht es darum, den Zusammenhang zwischen unterschiedlichen Werten zu verschleiern. Die Diffusion bewirkt den Einfluss
%eines Eingabebits auf viel Ausgabebits, um statistische Eigenschaften der Eingabe zu verstecken \cite[57]{crypto1}.

\subsection{Choose (CH)}
Choose steht für Wählen und beschreibt das Verhalten eines Multiplexers. Dabei bestimmt der Wert von x, ob y oder z das Ergebnis der Funktion bildet.
Durch die Datenwortbreite von 32 Bit, kommt es somit zu einer bitweisen Vermischung von y und z, die durch die Bits von x bestimmt wird. Dargestellt
ist die Funktion in Abbildung \ref{eq:ch}. Gemäß Standard \cite[10]{nist1804} führt das Ersetzen des XOR-Operators durch einen OR-Operator zu einem identischen Ergebnis.
\begin{figure}[!h]
  \begin{align}
  \text{CH}( x, y, z) &= (x~\wedge~y)~\oplus~( \neg~x~\wedge~z) \nonumber
  \end{align}
  \caption{Choose (CH)}
  \label{eq:ch}
\end{figure}

\subsection{Majority (MAJ)}
Majority steht für Mehrheit und bezieht sich auf die Anzahl der "`1"' und "`0"' Bits. Genau wie bei der Funktion "`Choose"' wird das Ergebnis aus drei Eingaben berechnet.
Es ergibt sich ein Ergebnis von "`1"', wenn mindestens zwei der Eingaben mit "`1"' belegt sind. Auch hier führt das Ersetzen des XOR-Operators durch einen OR-Operator zu
einem identischen Ergebnis.
\begin{figure}[!h]
  \begin{align}
  \text{MAJ}( x, y, z) &= (x~\wedge~y)~\oplus~(x~\wedge~z)~\oplus~(y~\wedge~z) \nonumber
  \end{align}
  \caption{Majority (MAJ)}
  \label{eq:maj}
\end{figure}

\subsection{Sigma (SIG)}
Die Sigma-Funktionen werden in den \acr{rfc}s der \acr{ietf} mit SSIG und BSIG beschrieben, das vermutlich aus "`Small Sigma"' und "`Big Sigma"' abgeleitet wurde
um Sonderzeichen zu vermeiden. Die Sigma-Funktionen werden auch als Substitutions-Boxen (S-Boxen) bezeichnet \cite[1]{sha256analyse}. Jedes Bit der Ausgabe wir dabei
aus zwei bis drei Eingabebits berechnet. Da die Funktion 32 Bit auf 32 Bit abbildet, werden somit alle Eingabebits für zwei bis drei Ausgabebits verwendet.
Dargestellt sind alle vier Sigma-Funktionen in Abbildung \ref{eq:sig}.

Innerhalb der Funktionen wird die Rotation nach rechts (ROTR) und die Verschiebung nach rechts (SHR) verwendet. Im Unterschied zur Rotation, bei der die Bits,
die aus dem Register geschoben werden, auf der anderen Seite wieder eingefügt werden, werden die Bits bei der Verschiebung mit "`$0$"'en aufgefüllt.

Während bei den $\Sigma$-Funktionen konsequent drei Eingabebits mit dem XOR-Operator verknüpft werden um ein Ausgabebit zu berechnen, werden bei den $\sigma$-Funktionen
einige Bits durch die Verschiebung verworfen, was einem Angreifer die Analyse erschwert.

\begin{figure}[!h]
  \begin{align}
  \sigma_0(x) &= \text{ROTR}^{7}(x)~\oplus~\text{ROTR}^{18}(x)~\oplus~\text{SHR}^{3}(x) \nonumber\\
  \sigma_1(x) &= \text{ROTR}^{17}(x)~\oplus~\text{ROTR}^{19}(x)~\oplus~\text{SHR}^{10}(x) \nonumber\\
  \nonumber\\
  \Sigma_0(x) &= \text{ROTR}^{2}(x)~\oplus~\text{ROTR}^{13}(x)~\oplus~\text{ROTR}^{22}(x) \nonumber\\
  \Sigma_1(x) &= \text{ROTR}^{6}(x)~\oplus~\text{ROTR}^{11}(x)~\oplus~\text{ROTR}^{25}(x) \nonumber
  \end{align}
  \caption{Sigma (SIG)}
  \label{eq:sig}
\end{figure}