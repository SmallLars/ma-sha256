\section{Distanzermittlung von Klauseln in Sha256}
\label{sec:ana:distance}

Die Analyse der neuen Klauseln aus der Kompressionsfunktion erfolgt auf Basis einer Distanzmetrik. Je höher die Distanz einer Klausel, desto höher
ist ihr angenommener Wert. Eine hohe Distanz bedeutet, dass die Klausel Literale aus Modulen in Zusammenhang zu bringt, die bisher in keiner direkten
Beziehung stehen.

Für die Distanzberechnung wird ein ungerichteter Graph aus den Modulen mit Level 10 erstellt. Ein Modul definiert sich dabei aus zusätzliche Literalen
und den Ausgangsliteralen. Im Gegensatz zum letzten Abschnitt fallen die Eingangsliterale weg, um eine eindeutige Zuordnung eines Literals zu einem
Modul durchführen zu können. Realisiert wird der Graph durch einen Collector mit dem Namen "`ModulGraph"'. Genau wie die ModulDB aus dem letzten Abschnitt
überschreibt der ModulGraph nur die Methode newModul des Collectors, um die Registrierung der Module zu erfassen. Um die Zuordnung der Literale zu einem
Modul schnellstmöglich durchführen zu können, wird eine Lookup-Tabelle erstellt, in der zu jedem Literal das zugehörige Modul hinterlegt ist.


distanzberechnung der knoten mit dijkstra: einmal machen: lookup table
maximale dijkstra distanz ist 16: 15 module dazwischen

für anzahl der module: lookup table: literalnummer führt zu modul


stellt print funktion mit highlighting bereit: DOT-Format
Programm "`graphprinter"' nutzt print funktion



Das Programm "`distancechecker"'

distance = max\_distance - modulcount + 1

Ausgabe in eine DIMACS-Datei je distance

\TODO{erledigen}