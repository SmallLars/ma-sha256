\section{Erhaltene Klauseln}
\label{sec:ana:acquired}

Nach der mehrfachen iterativen Anwendung der Methodik aus den letzten Abschnitten, wird die Analyse beendet. Bei den letzten Iterationen konnten
keine weiteren Klauseln sinnvoller Länge für die Module gefunden werden. Anders ist die Situation bei Klauseln aus der Kompressionsfunktion mit
positiver Distanz. Nach wie vor werden Klauseln generiert, die bei gleicher Distanz tendenziell kürzer werden. Da aber bisher nicht bekannt ist,
ob diese Klauseln den Lösungsprozess tatsächlich beschleunigen, wird die Analyse beendet. Basierend auf der Bewertung der Klauseln in Kapitel
\ref{chp:bewertung} wird sich zeigen, ob die Fortsetzung der Suche nach Klauseln mit großer Distanz sinnvoll ist.

Tabelle \ref{fig:additional_clauses_add} zeigt zunächst die erhaltenen Klauseln für die zusätzlichen Komponenten des Addierers.
Die Halb- und Mod-X-Addierer sind jeweils einmal enthalten, während die Volladdierer mit zwei bis vier Bit 29 bis 27 Mal enthalten sind.
Dadurch ergeben sich für den Addierer selbst insgesamt die zusätzlichen Klauseln in der unteren Zeile.
\begin{table}[!h]
  \centering
  \begin{tabular}{l|R{1.5cm}R{1.5cm}R{1.5cm}R{1.5cm}R{1.5cm}}
    \hiderowcolors
          & \multicolumn{5}{c}{\textbf{Klausellänge}} \\
    \cline{2-6}
    \textbf{Modul} & \textbf{3} & \textbf{4} & \textbf{5} & \textbf{6} & \textbf{7} \\
    \hline
    \showrowcolors
    Add\_Half\_$2$ & 9 &  16 &    9 &     &    \\
    Add\_Full\_$2$ &   &  22 &   38 &     &    \\
    Add\_ModX\_$2$ &   &     &   24 &     &    \\
    Add\_Half\_$3$ &   &   4 &    9 &   6 &    \\
    Add\_Full\_$3$ &   &     &   17 &  33 &    \\
    Add\_ModX\_$3$ &   &     &      &  27 &    \\
    Add\_Half\_$4$ &   &     &    5 &   2 &  4 \\
    Add\_Full\_$4$ &   &     &      &     &    \\
    Add\_ModX\_$4$ &   &     &      &     & 16 \\
    \hline
    Add          & 9 & 658 & 1625 & 959 & 20 \\
  \end{tabular}
  \caption{Erworbene Klauseln im Addierer}
  \label{fig:additional_clauses_add}
\end{table}
Erkennen lässt sich, dass die Länge der Klauseln mit der Ebene der Komponenten zunehmen. Gleichzeitig nimmt die Anzahl der erhaltenen Klauseln ab.
Das lässt sich dadurch erklären, dass bspw. ein 3-Bit Volladdierer auf den Literalen arbeitet, auf denen bereits zwei 2-Bit Volladdierer und drei
Volladdierer vorhanden sind. Damit ist ein Großteil des Verhaltens bereits definiert und es werden nur zusätzliche Klauseln benötigt, die bisher
fehlendes Verhalten definieren.

Tabelle \ref{fig:additional_clauses_mod} zeigt, wie viele Klauseln mit welcher Anzahl Literale in den essentiellen Modulen gefunden werden konnten.
Dazu gehören sowohl normale Klauseln als auch XOR-Klauseln. Sind in einer Zelle zwei Werte aufgeführt, gibt der erste Wert die Anzahl der zusätzlichen
Klauseln bei einer XOR-Unterstützung an. Der zweite Werte gibt die Anzahl der zusätzlichen Klauseln ohne XOR-Unterstützung wieder.
\begin{table}[!h]
  \centering
  \begin{tabular}{l|R{1.5cm}R{1.5cm}R{1.5cm}R{1.5cm}R{1.5cm}R{1.5cm}}
    \hiderowcolors
          & \multicolumn{6}{c}{\textbf{Klausellänge}} \\
    \cline{2-7}
    \textbf{Modul} & \textbf{2} & \textbf{3} & \textbf{4} & \textbf{5} & \textbf{6} & \textbf{7} \\
    \hline
    \showrowcolors
    Add\_Half\_$1$       & 1 & 2 - 3 &         &          &     &    \\
    Add\_Full\_$1$       &   & 6 - 8 &   0 - 2 &          &     &    \\
    Add\_ModX\_$1$       &   &       &         &          &     &    \\
    Add                  &   &     9 &     658 &     1625 & 959 & 20 \\
    Ssig$0$ ($\sigma_0$) &   &       &         & 3 - ~~48 &     &    \\
    Ssig$1$ ($\sigma_1$) &   &       &         &  7 - 112 &     &    \\
    Bsig$0$ ($\Sigma_0$) &   &       &         &          &     &    \\
    Bsig$1$ ($\Sigma_1$) &   &       &         &          &     &    \\
    Choose               &   &    64 &         &          &     &    \\
    Majority             &   &       &         &          &     &    \\
    Add\_Ssig            &   &       &       8 &          &     &    \\
    Add\_B$0$Maj           &   &       &         &          &     &    \\
    Add\_B$1$Ch            &   &       &     244 &      122 &     &    \\
    Prepare              & 2 &     8 &      68 &          &     &    \\
    ShaCore              & 3 &    32 & 81 - 95 &        1 &     &    \\
  \end{tabular}
  \caption{Erworbene Klauseln in den Modulen}
  \label{fig:additional_clauses_mod}
\end{table}
Der Mod-2 Addierer verwendet, genau wie die $\Sigma$-Funktionen, ausschließlich den XOR-Operator. Weitere Klauseln waren darin deshalb nicht zu erwarten und
wurden auch nicht gefunden. Bei den $\sigma$-Funktionen konnten weitere Klauseln gefunden werden da bei diesen, im Gegensatz zu den $\Sigma$-Funktionen, der
Shift-Operator verwendet wird, durch den einige Bits verworfen werden. Das zeigt sich auch bei der Addition der Ergebnisse der $\sigma$-Funktionen, wo ebenfalls
einige Klauseln gefunden werden konnten.

Im Modul der Choose-Funktion wurden zwei weitere Klauseln gefunden, die sich auf die Breite von 32 Bit anwenden lassen. Dadurch ergeben sich
64 zusätzliche Klauseln. Im Gegensatz dazu konnten bei der Majority-Funktion keine weiteren Klauseln gefunden werden. Dieses Muster findet sich auch
in den Modulen wieder, in denen das Ergebnis der Choose- und Majority-Funktion mit den Ergebnissen der $\Sigma$-Funktionen addiert wird. Nur in dem
Modul, in dem die Choose-Funktion verwendet wird (Add\_B$1$Ch), konnten weitere Klauseln gefunden werden.

Erfolgreicher ist die Ausbeute bei der Erweiterung der Eingabe (Prepare) und der Rundenfunktion (ShaCore). In diesen Modulen konnten einige Klauseln gefunden
werden, die sich aus dem Zusammenspiel zwischen Modulen kleineren Levels ergeben.

In der Kompressionsfunktion wurden die Klauseln mit positiver Distanz aus Tabelle \ref{fig:additional_clauses} ermittelt.
Basis für die insgesamt 29703 Klauseln bildeten 1107 Klauseln aus CryptoMiniSat. 28596 Klauseln wurden durch die Verallgemeinerung
auf Bitbreite und Runden ermittelt.
\begin{table}[!h]
  \centering
  \begin{tabular}{r|R{1.5cm}R{1.5cm}R{1.5cm}R{1.5cm}R{1.5cm}|R{1.5cm}}
    \hiderowcolors
          & \multicolumn{5}{c}{\textbf{Klausellänge}} \\
    \cline{2-6}
    \textbf{Distanz} & \textbf{2} & \textbf{3} & \textbf{4} & \textbf{5} & \textbf{6} & $ \boldsymbol{\sum} $ \\
    \hline
    \showrowcolors
                        1 & 428 &  412 &       &      &      &   840 \\
                        2 &  50 & 5287 &       &      &      &  5337 \\
                        3 &     &  159 &  9560 & 1430 &      & 11149 \\
                        4 &     &  318 &  1829 & 2068 & 5787 & 10002 \\
                        5 &     &   28 &   188 & 1469 &  718 &  2403 \\
    \hline
    $ \boldsymbol{\sum} $ & 478 & 6204 & 11577 & 4967 & 6505 & 29731 \\
  \end{tabular}
  \caption{Erworbene Klauseln in der Kompressionsfunktion}
  \label{fig:additional_clauses}
\end{table}

Wie in Abschnitt \ref{sec:ana:subclauses} erwähnt sind von diesen Klauseln einige überflüssig, weil bereits Don't-Care Literale identifiziert wurden
und kürzere Klauseln vorhanden sind. Tabelle \ref{fig:additional_clauses_clean} zeigt die Anzahl der sinnvollen Klauseln. 8425 Klauseln haben sich
als überflüssig herausgestellt womit 21306 sinnvolle Klauseln verbleiben.
\begin{table}[!h]
  \centering
  \begin{tabular}{r|R{1.5cm}R{1.5cm}R{1.5cm}R{1.5cm}R{1.5cm}|R{1.5cm}}
    \hiderowcolors
          & \multicolumn{5}{c|}{\textbf{Klausellänge}} & \\
    \cline{2-6}
    \textbf{Distanz} & \textbf{2} & \textbf{3} & \textbf{4} & \textbf{5} & \textbf{6} & $ \boldsymbol{\sum} $ \\
    \hline
    \showrowcolors
                        1 & 336 &  390 &      &      &      &   726 \\
                        2 &  50 & 3812 &      &      &      &  3862 \\
                        3 &     &   41 & 6849 &  954 &      &  7844 \\
                        4 &     &      &  778 & 2004 & 3896 &  6678 \\
                        5 &     &      &   33 & 1452 &  711 &  2196 \\
    \hline
    $ \boldsymbol{\sum} $ & 386 & 4243 & 7660 & 4410 & 4607 & 21306 \\
  \end{tabular}
  \caption{Erworbene Klauseln in der Kompressionsfunktion nach Bereinigung}
  \label{fig:additional_clauses_clean}
\end{table}

Anzumerken ist, dass die Ermittlung von Klauseln mit positiver Distanz nicht abgeschlossen ist. Die bisher erhaltenen Klauseln werden in Kapitel 6
genutzt um zu prüfen ob sie den Lösungsprozess beschleunigen. Dabei wird sich zeigen ob es sinnvoll ist die Suche nach weiteren Klauseln zukünftig
voranzutreiben.