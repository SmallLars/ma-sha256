\section{Erhaltene Klauseln}
\label{sec:ana:acquired}

Nach der mehrfachen iterativen Anwendung der Methodik aus den letzten Abschnitten, wird die Analyse beendet. Bei den letzten Iterationen konnten
keine weiteren Klauseln sinnvoller Länge für die Module gefunden werden. Anders ist die Situation bei Klauseln aus der Kompressionsfunktion mit
positiver Distanz. Nach wie vor werden Klauseln generiert, die bei gleicher Distanz tendenziell kürzer werden. Da aber bisher nicht bekannt ist,
ob diese Klauseln den Lösungsprozess tatsächlich beschleunigen, wird die Analyse beendet. Basierend auf der Bewertung der Klauseln in Kapitel
\ref{chp:bewertung} wird sich zeigen, ob die Fortsetzung der Suche nach Klauseln mit großer Distanz sinnvoll ist.

Tabelle \ref{fig:additional_clauses_mod} zeigt, wie viele Klausel mit welcher Anzahl Literale in den Modulen gefunden werden konnten. Dazu gehören sowohl
sowohl normale Klauseln als auch XOR-Klausen. Sind in einer Zelle zwei Werte aufgeführt, gibt der erste Wert die Anzahl der zusätzlichen Klauseln bei einer
XOR-Unterstützung an. Der zweite Werte gibt die Anzahl der zusätzlichen Klauseln ohne XOR-Unterstützung wieder.
\begin{table}[!h]
  \centering
  \begin{tabular}{l|R{1.5cm}R{1.5cm}R{1.5cm}R{1.5cm}R{1.5cm}R{1.5cm}}
    \hiderowcolors
          & \multicolumn{6}{c}{\textbf{Klausellänge}} \\
    \cline{2-7}
    \textbf{Modul} & \textbf{2} & \textbf{3} & \textbf{4} & \textbf{5} & \textbf{6} & \textbf{7} \\
    \hline
    \showrowcolors
    Add\_Half\_1 & 1 & 2 - 3 &         &          &     &    \\
    Add\_Full\_1 &   & 6 - 8 &   0 - 2 &          &     &    \\
    Add\_Last\_1 &   &       &         &          &     &    \\
    Add          &   &     9 &     658 &     1625 & 959 & 20 \\
    Ssig0        &   &       &         & 3 - ~~48 &     &    \\
    Ssig1        &   &       &         &  7 - 112 &     &    \\
    Bsig0        &   &       &         &          &     &    \\
    Bsig1        &   &       &         &          &     &    \\
    Ch           &   &    64 &         &          &     &    \\
    Maj          &   &       &         &          &     &    \\
    Add\_Ssig    &   &       &       8 &          &     &    \\
    Add\_B0Maj   &   &       &         &          &     &    \\
    Add\_B1Ch    &   &       &     244 &      122 &     &    \\
    Prepare      & 2 &     8 &      68 &          &     &    \\
    ShaCore      & 3 &    32 & 81 - 95 &        1 &     &    \\
  \end{tabular}
  \caption{Erworbene Klauseln in den Modulen}
  \label{fig:additional_clauses_mod}
\end{table}

Die zusätzlichen Klauseln der Addierer höherer Ebene sind zunächst im Addierer zusammengefasst. Die Halb- und Mod-2-Addierer sind jeweils einmal enthalten,
während die Volladdierer mit zwei bis vier Bit 29 bis 27 Mal enthalten sind. Da der Mod-2 Addierer \TODO{XOR - keine weiteren klauseln}

Tabelle \ref{fig:additional_clauses_add} zeigt die gefundenen Klauseln für die Addierer höherer Ebene im Detail.
\begin{table}[!h]
  \centering
  \begin{tabular}{l|R{1.5cm}R{1.5cm}R{1.5cm}R{1.5cm}R{1.5cm}}
    \hiderowcolors
          & \multicolumn{5}{c}{\textbf{Klausellänge}} \\
    \cline{2-6}
    \textbf{Modul} & \textbf{3} & \textbf{4} & \textbf{5} & \textbf{6} & \textbf{7} \\
    \hline
    \showrowcolors
    Add\_Half\_2 & 9 & 16 &  9 &    &    \\
    Add\_Full\_2 &   & 22 & 38 &    &    \\
    Add\_Last\_2 &   &    & 24 &    &    \\
    Add\_Half\_3 &   &  4 &  9 &  6 &    \\
    Add\_Full\_3 &   &    & 17 & 33 &    \\
    Add\_Last\_3 &   &    &    & 27 &    \\
    Add\_Half\_4 &   &    &  5 &  2 &  4 \\
    Add\_Full\_4 &   &    &    &    &    \\
    Add\_Last\_4 &   &    &    &    & 16 \\
  \end{tabular}
  \caption{Erworbene Klauseln im Addierer}
  \label{fig:additional_clauses_add}
\end{table}

\begin{table}[!h]
  \centering
  \begin{tabular}{r|R{1.5cm}R{1.5cm}R{1.5cm}R{1.5cm}R{1.5cm}|R{1.5cm}}
    \hiderowcolors
          & \multicolumn{5}{c}{\textbf{Klausellänge}} \\
    \cline{2-6}
    \textbf{Distanz} & \textbf{2} & \textbf{3} & \textbf{4} & \textbf{5} & \textbf{6} & $ \boldsymbol{\sum} $ \\
    \hline
    \showrowcolors
                        1 & 428 &  412 &       &      &      &   840 \\
                        2 &  50 & 5287 &       &      &      &  5337 \\
                        3 &     &  159 &  9560 & 1430 &      & 11149 \\
                        4 &     &  318 &  1829 & 2068 & 5787 & 10002 \\
                        5 &     &   28 &   188 & 1469 &  718 &  2403 \\
    \hline
    $ \boldsymbol{\sum} $ & 478 & 6204 & 11577 & 4967 & 6505 & 29731 \\
  \end{tabular}
  \caption{Erworbene Klauseln in der Kompressionsfunktion}
  \label{fig:additional_clauses}
\end{table}

\begin{table}[!h]
  \centering
  \begin{tabular}{r|R{1.5cm}R{1.5cm}R{1.5cm}R{1.5cm}R{1.5cm}|R{1.5cm}}
    \hiderowcolors
          & \multicolumn{5}{c|}{\textbf{Klausellänge}} & \\
    \cline{2-6}
    \textbf{Distanz} & \textbf{2} & \textbf{3} & \textbf{4} & \textbf{5} & \textbf{6} & $ \boldsymbol{\sum} $ \\
    \hline
    \showrowcolors
                        1 & 336 &  390 &      &      &      &   726 \\
                        2 &  50 & 3812 &      &      &      &  3862 \\
                        3 &     &   41 & 6849 &  954 &      &  7844 \\
                        4 &     &      &  778 & 2004 & 3896 &  6678 \\
                        5 &     &      &   33 & 1452 &  711 &  2196 \\
    \hline
    $ \boldsymbol{\sum} $ & 386 & 4243 & 7660 & 4410 & 4607 & 21306 \\
  \end{tabular}
  \caption{Erworbene Klauseln in der Kompressionsfunktion nach Bereinigung}
  \label{fig:additional_clauses_clean}
\end{table}

\TODO{erledigen}