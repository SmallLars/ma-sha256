\section{Subklauselprüfung}
\label{sec:ana:subclauses}

Durch die mehrfache iterative Anwendung der Methodik aus den letzten Abschnitten, kommt es zu einer kontinuierlichen Erweiterung der Klauselmenge.
Dazu zählt auch, dass Dont-Care Literale in Klauseln identifiziert und entfernt werden. Wenn die neuen kürzeren Versionen dieser Klauseln durch die
Analyse in die Klauselmenge aufgenommen werden, sind die alten langen Klauseln überflüssig. Um diese zu erkennen wird das Programm "`clausesorter"'
genutzt. Dieses bekommt als Eingabe alle bekannten Klauseln und überprüft diese auf kürzere Klauseln. Die naive Variante hat eine quadratische
Laufzeit im Bezug zur Eingabegröße, da zu jeder Klausel alle weiteren überprüft werden müssen. Um die Laufzeit zu reduzieren wird eine Lookup-Tabelle
genutzt, in der zu jedem Literal die Klauseln hinterlegt sind, die das Literal enthalten. So werden zu jeder Klausel nur noch die Klauseln überprüft,
die mindestens ein gemeinsames Literal haben. In der Praxis reduziert sich die Laufzeit bei zwei Millionen Klausel dadurch von einer Woche auf fünf Minuten.

Die als überflüssig erkannten Klauseln werden in eine separate Datei ausgegeben und genau wie neue Klauseln einem Modul zugeordnet. Dadurch können
diese Klauseln in den Modulen erkannt und entfernt werden. Zu Dokumentationszwecken werden die Klauseln nur auskommentiert. Eine Ausnahme bildet
die Kompressionsfunktion in der die Klauseln mit positiver Distanz enthalten sind. Der Aufwand die überflüssige Klausel im Programmcode zu identifizieren
ist bei geringem Nutzen vergleichsweise hoch. Das Modul, und somit jede enthaltene Klausel, wird nur einmal verwendet. Die Anzahl der überflüssigen
Klauseln ist damit am Ende verhältnismäßig gering und sollte CryptoMiniSat nicht bremsen.
