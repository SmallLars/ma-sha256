\section{Aussortieren bekannter und doppelter Klauseln}
\label{sec:ana:rem_double}

Da die Entscheidung, ob eine XOR-Unterstützung vorliegt, beim Kompilieren getroffen werden muss, werden die bekannten Klauseln
mit dem Programm "`dimacsprinter"' nach jedem Lösungsversuch und der darauf folgenden Analyse in zwei Dateien im DIMACS-Format
ausgegeben. Die eine Datei enthält dabei die Klauselmenge ohne XOR-Unterstützung, während die andere Datei die Klauselmenge für
eine XOR-Unterstützung enthält. Das Programm "`clausecollector"' liest beide Dateien beim Start ein und führt die Klauselmengen
zusammen. Da CryptoMiniSat XOR-Klauseln akzeptiert, diese jedoch intern in normale Klauseln umrechnet, enthalten auch die
extrahierten Klauseln keine XOR-Klauseln. Der Parser für das DIMACS-Format konvertiert deshalb ebenfalls XOR-Klauseln in normale
Klauseln.

Abgelegt werden die bekannten Klauseln in einem "`set"'. Diese Datenstruktur ermöglicht die Suche nach einer Klausel in logarithmischer
Zeit im Bezug zur Anzahl der darin enthaltenen Klauseln. So kann effizient geprüft werden, ob eine extrahierte Klausel schon bekannt ist.
Neue Klauseln aus der irredundanten und der redundanten Klauselmenge werden ebenfalls jeweils in einem "`set"' gesammelt, wodurch die
Klauselmenge automatisch sortiert wird und doppelte Klauseln wegfallen. Abschließend werden die neuen Klauseln aus der irredundanten und
der redundanten Klauselmenge jeweils in eine DIMACS-Datei zur weiteren Analyse geschrieben.