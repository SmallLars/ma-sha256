\section{Vergleich der Resultate}

In allen Tests hat die XOR-Variante besser abgeschnitten als die Variante ohne XOR-Klauseln. Im Folgenden wird deshalb nur noch die XOR-Variante betrachtet.
Das beste Ergebnis haben die modulspezifischen Klauseln mit einer Reduktion der Testlaufzeit um fast 60\% erzielt. Auch die Distanzklauseln und das zusätzliche
Wissen haben die Testlaufzeit um 20\% bzw. 40\% reduziert. Negativ haben sich nur die zusätzlichen Addiererklauseln ausgewirkt. Die Vereinigung der positiv
getestet Klauseln hat nur eine Reduktion der Testlaufzeit um 20\% ergeben und konnte damit nicht an die 60\% der modulspezifischen Klauseln anknüpfen. Das ist
deshalb ein Problem, weil die modulspezifischen Klauseln mehr oder weniger "`abgeschlossen"' sind. Selbst wenn es in dieser Arbeit nicht gelungen ist, die
optimale "`vollständige"' Klauselmenge für die Module zu finden, ist diese doch begrenzt. Anders ist dies insbesondere bei den Distanzklauseln, da bei diesen
noch die Möglichkeit besteht viele weitere Klauseln zu ermitteln. Aus diesem Grund wird die Klauselmenge aus Abschnitt \ref{sec:test_beste} für die Evaluation
in Kapitel \ref{chp:evaluation} herangezogen. Gelingt es dieser Klauselmenge gegen andere Implementierungen zu bestehen, ist eine weitere Suche nach
Distanzklauseln möglicherweise sinnvoll.
