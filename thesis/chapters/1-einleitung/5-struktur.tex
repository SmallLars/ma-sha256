\section{Struktur}

Im Anschluss an die Einleitung folgt in \textbf{Kapitel \ref{chp:grundlagen}} eine kurze Erläuterung der
grundlegenden Dinge und Hilfsmittel die in dieser Arbeit verwendet werden. 

Es folgt in \textbf{Kapitel \ref{chp:sha256}} eine Beschreibung des Hash-Algorithmus \glos{sha256} um die relevanten
Komponenten für die Erzeugung der konjunktiven Normalform in \textbf{Kapitel \ref{chp:knf}} kennen zu lernen.

\textbf{Kapitel \ref{chp:analyse}} beschreibt die Analyse mit CryptoMiniSat und fasst die erhaltenen Klauseln
zusammen die in \textbf{Kapitel \ref{chp:bewertung}} einer Bewertung unterzogen werden.

\textbf{Kapitel \ref{chp:evaluation}} beinhaltet die Evaluation, in der das Ergebnis dieser Arbeit mit den
Arbeiten von Martin Maurer und Jonathan Heusser verglichen wird und eine Aussage darüber, ob praktisch
relevante Angriffe möglich sind.

\textbf{Kapitel \ref{chp:fazit}} enthält die persönliche Meinung des Autors.