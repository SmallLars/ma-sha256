\section{Ziel dieser Arbeit}
\label{sec:ziel}

Die im vorhergehenden Abschnitt genannten Arbeiten beschränken sich zu einem Großteil darauf, eine Eingabe für
einen SAT-Solver zu generieren und Versuche damit durchzuführen um die Laufzeiten zu ermitteln und zu vergleichen.
Vegard Nossum geht noch einen Schritt weiter und versucht das Wissen, das ein SAT-Solver während eines Lösungsversuchs
erwirbt, zu extrahieren und für weitere Lösungsversuche zu nutzen. Dieser Schritt erfolgt jedoch relativ kurz am Ende
seiner Arbeit und führt zum Ergebnis, dass dieses Wissen in weiteren Lösungsversuchen keinen Vorteil mit sich bringt.

Ziel dieser Arbeit ist es, diesen Schritt zu erweitern und das erworbene Wissen einer Analyse zu unterziehen um den
Lösungsprozess zu beschleunigen. Dabei sollen potenziell relevante Teile heraus gefiltert, und in weitere Lösungsversuche
eingebracht werden. Außerdem soll das erworbene Wissen, soweit möglich, verallgemeinert werden, um es auch an anderen Stellen,
die das gleiche Verhalten beschreiben, zu nutzen. Schließlich soll eine Aussage darüber getroffen werden, in wie fern die Dauer
der Berechnung eine praktische Relevanz für einen Angriff bietet.