\section{Ziel dieser Arbeit}

Die im vorhergehenden Abschnitt genannten Arbeiten beschränken sich zu einem Großteil darauf, eine Eingabe für
einen SAT-Solver zu generieren und Versuche damit durchzuführen um die Laufzeiten zu ermitteln und zu vergleichen.
Vegard Nossum geht noch einen Schritt weiter und versucht das Wissen, das ein SAT-Solver während eines Lösungsversuchs
erwirbt, zu extrahieren und für weitere Lösungsversuche zu nutzen. Dieser Schritt erfolgt jedoch relativ kurz am Ende
seiner Arbeit und führt zum Ergebnis, dass dieses Wissen in weiteren Lösungsversuchen keinen Vorteil mit sich bringt.
An diesem Punkt setzt diese Masterarbeit an um die Analyse des erworbenen Wissens zu erweitern und möglicherweise doch
einen Vorteil daraus zu ziehen. Dazu wird ein Hash-Algorithmus herangezogen und abschließend geprüft, ob die Berechnung
schnell genug erfolgt, um praktisch relevante Angriffe durchzuführen.