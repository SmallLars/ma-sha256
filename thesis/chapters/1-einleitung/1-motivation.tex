\section{Motivation}

Während meines Studiums an der Universität Bremen habe ich einige Vorlesungen der AG Rechnerarchitektur besucht.
In vielen der dabei durchgeführten Übungsaufgaben wurden Sat-Solver (siehe Abschnitt \ref{sec:satsolver}) eingesetzt
um eine Lösung für ein Problem zu finden. Im Fokus der Aufgaben stand die Kodierung eines Problems in ein für die
SAT-Solver geeignetes Format und die Auswertung der Lösung. Naturgemäß sind die Probleme der Übungsaufgaben keine
echte Herausforderung für einen SAT-Solver und innerhalb kürzester Zeit lösbar, selbst wenn die Kodierung nicht optimal
ist. Das hat mein Interesse geweckt mich mit "`richtigen"' Problemen zu beschäftigen und die Grenzen der Sat-Solver
zu erkunden.

Durch einige Vorlesungen der AG Rechnernetze und das dort absolvierte Bachelor/Master-Projekt zeigten sich Hash-Algorithmen
als ein interessantes Problem. Diese zeichnen sich dadurch aus, dass die Berechnung eines Hash zwar einfach, aber nicht
umkehrbar ist. Verwendet werden sie in vielen sicherheitskritischen Umgebungen, in denen der Hash als eine Art Fingerabdruck
von Daten genutzt wird, um eine Manipulation auszuschließen. Dazu gehört auch die digitale Währung "`Bitcoin"' (siehe Abschnitt
\ref{sec:bitcoin}), in der sich durch "`Mining"' Geld verdienen lässt. Mining bedeutet in diesem Fall so lange Fingerabdrücke
zu erzeugen, bis ein für das System passender Fingerabdruck dabei herauskommt. Eine Umkehrung der Berechnung würde diesen
Prozess wesentlich einfacher machen.

Andere Arbeiten zu diesem Thema (siehe Abschnitt Abschnitt \ref{sec:otherwork}) haben bereits einen Versuch unternommen,
Hash-Berechnungen mit Sat-Solvern umzukehren. Diese Arbeiten beschränken sich jedoch darauf, eine mehr oder weniger optimierte
Eingabe für Sat-Solver zu erzeugen und diese mit verschiedenen Sat-Solvern zu testen. Im besten Fall werden außerdem noch
einige Versuche zu unternommen, die optimalen Einstellungen spezifischer Sat-Solver zu ermitteln.

An diesem Punkt setzt die Arbeit an und geht einen Schritt weiter. Während eines Lösungsversuch erwerben Sat-Solver zusätzliches
Wissen über das gestellte Problem. Im Allgemeinen wird dieses Wissen bei Programmende verworfen. In diesem Fall wird das Wissen
analysiert und von Anfang an in weitere Lösungsversuche integriert.

Als spezifischer Hash-Algorithmus wird SHA256 (siehe Kapitel \ref{chp:sha256}) verwendet. Dieser wird bei Bitcoin eingesetzt und
zeichnet sich dadurch aus, dass er einen Fingerabdruck in 64 ähnlichen Schritten berechnet. Wissen das über einen Schritt
erworben wird, kann so unter Umständen für die anderen 63 Schritte übernommen werden.