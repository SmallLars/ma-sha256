\section{Vorgehensweise}

Als Hash-Algorithmus wird \glos{sha256} (siehe Kapitel \ref{chp:sha256}) ausgewählt. Dieser wird bei Bitcoin eingesetzt
und zeichnet sich dadurch aus, dass er einen Fingerabdruck in 64 ähnlichen Schritten berechnet. Wissen das über einen Schritt
erworben wird, kann so unter Umständen für die anderen 63 Schritte übernommen werden. Außerdem haben Martin Maurer und Jonathan
Heusser \glos{sha256} in ihren Arbeiten verwendet, so dass abschließend ein Vergleich der Ergebnisse erfolgen kann. Der von Vegard Nossum
verwendete Hash-Algorithmus SHA1 berechnet einen Fingerabdruck in 80 Schritten, wobei vier unterschiedliche Schritte jeweils 20 Mal
zum Einsatz kommen.

Verwendet wird in dieser Arbeit ausschließlich der SAT-Solver CryptoMiniSat. Dieser zeichnet sich dadurch aus, dass er nicht nur
eine konjunktive Normalform als Eingabe akzeptiert, sondern auch in der Lage ist Klauseln im XOR-Format zu verarbeiten. Eine Klausel
ist dabei eine Verknüpfung von Werten mit dem OR-Operator. Klauseln im XOR-Format verwenden den XOR-Operator. Dies ermöglicht
eine kompaktere Repräsentation der Eingabe und beschleunigt bei richtiger Verwendung den Lösungsprozess des SAT-Solvers.
Außerdem sammelt CryptoMiniSat nicht nur zusätzliches Wissen, sondern versucht auch die eingegebene konjunktive Normalform zu
optimieren. Sowohl die optimierte Eingabe, als auch das zusätzliche Wissen lassen sich jederzeit einfach abrufen und analysieren.

Um eine Analyse der Daten durchzuführen, die CryptoMiniSat während eines Lösungsversuchs sammelt, reicht es nicht aus \glos{sha256}
in eine konjunktive Normalform zu überführen und an CryptoMiniSat zu übergeben. Dieser Ansatz wird von Martin Maurer und Jonathan
Heusser verfolgt. Jedoch ermöglicht er es nicht, von CryptoMiniSat erworbenes Wissen über die konjunktive Normalform den
entsprechenden Teilen von \glos{sha256} zuzuordnen. Für diesen Prozess müssen während der Erzeugung der konjunktiven Normalform
Daten darüber gesammelt werden, die eine Zuordnung ermöglichen. Zu diesem Zweck wird ein Framework erstellt, das je nach Bedarf
die konjunktive Normalform erzeugt und gleichzeitig die dazugehörigen Daten zur Verfügung stellt. Mit diesem Framework wird
im Anschluss eine konjunktiven Normalform erzeugt, die mit denen von Martin Maurer und Jonathan Heusser
verglichen wird, um diese Basis zu überprüfen und im Zweifelsfall noch weiter zu optimieren.

Mit der erstellten konjunktiven Normalform werden iterativ mehrere Lösungsversuche durchgeführt. Die dafür verwendeten
Daten werden separat als "`Annahmen"' an CryptoMiniSat übergeben. Das ermöglicht eine Trennung zwischen dem Wissen, was
auf den Daten basiert, und dem Wissen was ausschließlich auf der konjunktiven Normalform basiert. Nur Letzteres ist für
diese Arbeit relevant, da dieses Wissen für \glos{sha256} im Allgemeinen gilt. Für die Auswertung dieses Wissens werden einige
Hilfsprogramme erstellt, die potentiell interessante Klauseln filtern und, basierend auf den gesammelten Informationen,
sortieren. Diese zusätzlichen Klauseln werden im Anschluss darauf geprüft, ob sie den Lösungsprozess beschleunigen.

Abschließend wird die konjunktive Normalform inklusive der zusätzlichen Klauseln mit der Performance der Implementierungen
von Martin Maurer und Jonathan Heusser verglichen. Außerdem wird überprüft, ob die Berechnung schnell genug erfolgt um
praktisch relevante Angriffe auf \glos{sha256} und Bitcoin durchzuführen.